
\documentclass[11pt]{article}
\usepackage[margin=.8in]{geometry}
\usepackage{amsmath,amssymb,amsthm, latexsym, mathrsfs, pdfsync, multicol,
%setspace,
%graphics, 
fancybox, fancyhdr,
graphicx, enumerate,
subfig, tikz, pgfplots,array,tabularx}
\usepackage{bbding}

%\singlespacing
\def\RR{{\mathbb R}}
\def\NN{{\mathbb N}}
\def\ZZ{{\mathbb Z}}
\def\QQ{{\mathbb Q}}
\def\CC{{\mathbb C}}
\def\bc{\begin{center}}
\def\ec{\end{center}}
\def\be{\begin{enumerate}}
\def\ee{\end{enumerate}}
\def\bi{\begin{itemize}}
\def\ei{\end{itemize}}
\def\bs{\begin{slide}}
\def\es{\end{slide}}
\def\bx{\begin{exercise}}
\def\ex{\end{exercise}}
\def\t{\times}
\newcommand{\ol}[1]{\overline{#1}}
\newcommand{\oimp}[1]{\overset{#1}{\Longleftrightarrow}}
\newcommand{\bv}[1]{\ensuremath{ \mathbf{\vec{#1}}} }
\renewcommand{\d}{\displaystyle}
\newcommand{\blank}[1]{\rule{#1}{0.75pt}}

\usetikzlibrary{calc}

%for tikz pictures
\pgfplotsset{compat=1.6}

\pgfplotsset{soldot/.style={color=black,only marks,mark=*}} \pgfplotsset{holdot/.style={color=black,fill=white,only marks,mark=*}}
\pgfplotsset{my style/.append style={axis x line=middle, axis y line=middle, xlabel={$x$}, ylabel={$y$}}}


%
% Answerbox:
%
%\newcommand\answerbox[3]{#3 \fbox{\rule{#1}{0cm}\rule{0cm}{#2}}}
%
%\setlength{\headsep}{2pt}

\lhead{\sc{Math 251 Calculus I}}
\chead{\large \sc Final Exam} 
\rhead{\sc Fall 2019}
\cfoot{}
\pagestyle{fancy}
%
\begin{document}
\thispagestyle{fancy}

\vspace{.1in}
\begin{tabular}{l@{\hspace{.4in}}l}
Your Name & Your Signature\\
\framebox(200,30){} & \framebox(200,30){} \\
\end{tabular}

%\bigskip

\begin{tabular}{l@{\hspace{.4in}}l}
Instructor Name & \\
\framebox(200,30){}&  \\
\end{tabular}
{
\renewcommand{\baselinestretch}{1.8}
\setlength{\tabcolsep}{.2in}
\normalsize
\begin{center}
\begin{tabular}{|c|c|c|}
\hline
Problem&Total Points&\parbox{.8in}{\hfil Score\hfil}\\
\hline
1&10&\\
\hline
2&10&\\
\hline
3&10&\\
\hline
4&10&\\
\hline
5&10&\\
\hline
6&10&\\
\hline
7&10&\\
\hline
8&10&\\
\hline
9&10&\\
\hline
10&10&\\
\hline
\hline
Extra Credit & (5) & \\
%\hline
\hline
Total&100&\\
\hline
%Current Course Grade&\multicolumn{2}{c|  }{}\\
%\hline

\end{tabular}

\end{center}
}
\begin{itemize}
\item 
This test is closed notes and closed book. 

\item You may \textbf{not} use a calculator.

\item
In order to receive full credit, you must {\bf show your work.}  
Be wary of doing computations in your head. Instead, write out your
computations on the exam paper.
 
\item
Raise your hand if you have a question.

\end{itemize}

\newpage
\vspace*{-0.3in}
\begin{enumerate}
%Definition of the derivative and equation of tangent line
\item (10 points) Let $f(x)=x^2-\frac{3}{x}.$
	\begin{enumerate}
	\item Using the \textbf{definition of the derivative}, find $f'(x)$. No credit will be given if a different method is used. [It is recommended you start by writing the definition of the derivative.] 
	\vfill
	\item Write an equation of the line tangent to the graph of $f(x)$ when $x=2.$	
	\vspace{2in}
	\end{enumerate}
\newpage
%net change
\item (10 points) During a storm, snow is falling on a mountain at a rate of $$M(t)= t^2-\frac{t^3}{3}$$ feet per hour for a three hour period starting at time $t=0.$
	\begin{enumerate}
	\item Determine the \emph{net change} in the height of snow during the first two hours of the storm. Include  units with your answer.
	\vfill
	\item Determine the height of the snow on the mountain or explain why this is not possible with the present information.
	\vspace{2in}
	\item Observe that $M(2.5)>0$ and $M'(2.5)<0.$ Explain what these two facts indicate about the snow falling when $t=2.5.$
	\vspace{1in}
	\end{enumerate}
\newpage
%interpret derivatives and limits
\item (10 points) The population of ants in a new colony is modeled by the function
$$p(t)=1000\left(t+\frac{1}{2} \ln(1+t^2)\right)+100,$$ where $t$ is measured in months. 
	\begin{enumerate}
	\item Find $p(0)$ and interpret in the context of the problem. \\
	\vfill
	 \item Find $\displaystyle \lim_{t \to \infty} p(t)$ and interpret in the context of the problem. \\
	\vfill
	\item Find $p'(t).$ \\
	\vfill
	\item Find $p'(1)$ and interpret in the context of the problem. \\
	\vfill
	\item  Find $\displaystyle \lim_{t \to \infty} p'(t)$ and interpret in the context of the problem. \\
	\vfill
	\end{enumerate}
\newpage
%given pic, answer questions
\item (10 points) Consider the function $A(x)$ graphed below. Between $x=3$ and $x=5$, the graph is of a semicircle of radius 1.

\begin{center}
\begin{tikzpicture}[scale=1]
\draw[<->] (-3.2,0) -- (6.2,0) node[right] {$x$};
\draw[<->] (0,-2.2) -- (0,3.2); %node[left] {$A(x)$};
\draw (-.5, 1.5) node{$A(x)$};
\draw[help lines, dashed] (-3,-2) grid (6,3);
\draw[style= ultra thick,<-] (-3.2,3.2) --  (1,-1) -- (2,-1)--(3,0);
\draw[style= ultra thick,->] (3,0) arc (180:0:1) -- (6.2,1.1);
\foreach \x in {-3,-2,-1,0,1,2,3,4,5,6}
\draw (\x,-2.1) node[below] {$\x$};
\foreach \y in {-1,0,1,2}
\draw (-3.1,\y) node[left] {$\y$};
\node[fill=white, minimum size=1pt,circle,draw,scale=.55] at (-2,2){};
\node[fill=black, minimum size=1pt,circle,draw,scale=.5] at (-2,1){};
\end{tikzpicture}
\end{center}

\bigskip
\begin{enumerate}
\item  $\displaystyle \lim_{x \to -2} A(x)=$\\

\item  $A(-2)=$\\

\item  $A'(-1)=$\\

\item At what $x$ values, if any, does $A'(x)$ not exist?
\vfill


\item Evaluate $\displaystyle \int_{-1}^{2} A(x)\,dx$.

\vfill

\item Let $H(x) =\displaystyle \int_0^x A(s)\,ds$.  What is the value of $H(4)$?

\vfill

\item For $H(x)$ from part \textbf{f.}, what is the value of $H'(4)$.

\vfill
\end{enumerate}
\newpage
%sketch the graph given the information
\item (10 points) Sketch the graph of a function that satisfies \textbf{all} of the given conditions. \textbf{Label all important items in your graph.}
	\begin{enumerate}
	\item The domain of $f$ is $(-\infty,\infty).$
	\item $f'(x) > 0$ if $x \not = 2.$\\
	\item $f''(x) > 0$ if $x < 2$ and $f''(x) <0$ if $x>2.$\\
	\item $f(2)=5$\\
	\item $\displaystyle \lim_{x \to \infty} f(x)=8$ and $\displaystyle \lim_{x \to -\infty} f(x)=0$\\
	\end{enumerate}
	\begin{center}
\begin{tikzpicture}
\draw[<->] (-6.2,0) -- (6.2,0) node[right] {$x$};
\draw[<->] (0,-6.2) -- (0,8.2) node[left] {$y$};
\end{tikzpicture}
	\end{center}
\newpage
%%%Plane-jane derivatives
\item (10 points) Differentiate the following functions. For parts (a) and (b), it is not necessary to simplify your answers.

\begin{enumerate}[(a)]
	\item $\d f(x) = \frac{e^{2x}}{\sqrt{x^2 + 1}} + \arctan(3x)$
	\vfill
	\item $\d g(x) = \int_x^2 t\cos(2t^2)\,dt$
	\vspace{1.5in}
	\item Find $\frac{dy}{dx}$ by implicit differentiation: $\d \ln(xy) - \cos y = ye^x$
	\vspace{3.5in}
	
	
\end{enumerate}

\newpage


%%%Optimization or min/max
\item (10 points) A piece of wire 40 cm long is to be cut to make at most two squares. See the picture below.

\begin{center}
\begin{tikzpicture}
\draw[ultra thick] (0,0) -- (10,0);
\draw (3.5,-0.2) -- (3.5,.2) node[above]{cut}; %node{\Large{\rotatebox[]{270}{\ScissorHollowLeft}}};
\node at (1.75,0.2){\large{$x$}};
%\node at (1.4,-0.4){\large{$a$}};
%\node at (2.4,-0.4){\rotatebox[]{270}{\ScissorHollowLeft}};
%\node at (-.4,2){\large{$b$}};
\draw[dashed] (0,-1) -- (4.5,-1);
\draw[dashed] (5.5,-1) -- (10,-1);
\draw[dashed] (0,-.7) -- (0,-1.3);
\draw[dashed] (10,-.7) -- (10,-1.3);
\node at (5,-1){\large{40cm}};
\end{tikzpicture}
\end{center}

\begin{enumerate}[(a)]
	\item Write the combined area of the two squares as a function of $x$. State the domain.
	\vfill
	\item For what value(s) of $x$ does the area function have a potential maximum or minimum?
	\vfill
	\item Where should you cut to minimize the combined area? Maximize the combined area? Justify the classification of the extrema.
	\vfill
\end{enumerate}

\newpage
%\item (6 points) Use \textbf{limits} to determine all holes and all horizontal and vertical asymptotes of the function
%\[ k(x) = \frac{x^2-4}{x^2+9 x+14}.\]
%
%\vfill

%\fbox{Version 2}
%
%Use \textbf{limits} to determine all holes and all horizontal and vertical asymptotes of the function
%\[ k(x) = \frac{x^2-9}{x^2+8 x+15}.\]
%
%\vfill


%\newpage
%	PROBLEM Definition of the derivative
%\newpage
%\item
%
%%\fbox{Version 1} 
%
%Suppose $f(x) = x^2 - \frac{3}{x}$
%\be
%\item Use the DEFINITION of the derivative to calculate $f'(x)$. (You should be using limits!)
%\vfill
%\vfill
%\vfill
%
%\item Write the equation for the tangent line to $f(x)$ at the point $(3, f(3))$.
%\vfill
%\ee
%
%\fbox{Version 2} 
%
%Suppose $f(x) = x^2 - \frac{5}{x}$
%\be
%\item Use the DEFINITION of the derivative to calculate $f'(x)$. (You should be using limits!)
%\vfill
%\vfill
%\vfill
%
%\item Write the equation for the tangent line to $f(x)$ at the point $(5, f(5))$.
%\vfill
%\ee

\newpage




%	PROBLEM II --- Antiderivatives
\item (10 points)

%\fbox{Version 1}

A particle is moving with acceleration
\[a(t) = t+e^{t/2}.\]
You measure that at time $t = 0$, its position is given by $s(0) = 3$ and its velocity is given by $v(0) = 8$.

Determine the position function $s(t)$. 
\vfill

%\fbox{Version 2}
%
%A particle is moving with acceleration
%\[a(t) = t+e^{t/2}.\]
%You measure that at time $t = 0$, its position is given by $s(0) = 2$ and its velocity is given by $v(0) = 5$.
%
%Determine the position function $s(t)$. 
%\vfill
%
%
\newpage
\item (10 points)
%\fbox{Version 1}

%Consider the function $g(x) = (2x-1)^{2} - 5$, part of whose graph is shown below.
%
%\begin{center}
%\begin{tikzpicture}[xscale=2, yscale = .5]
%\fill [gray!20, domain=0:2, variable=\x]
%  (0, 0)
%  -- plot ({\x}, {(2*\x-1)*(2*\x-1)-5})
%  -- (2, 0)
%  -- cycle;
%
%\draw[<->](-.1, 0) -- (2.3,0);
%\draw[<->] (0, -5.1) -- (0, 4);
%\foreach \i in {1,2,3} {\draw( 2/3*\i, -.1) -- (2/3*\i, .1);}
%\draw  (2/3*3, -.1) node[below] {2};
%\draw  (2/3*2, -20/9-1) node[below] {$g(x)$};
%%\draw[ultra thick, smooth, domain=0:??4??] plot ({\x, (2*\x-4)**2 - 2});
%
%
%  \draw[ultra thick, ] plot[smooth, domain=0:2] (\x,{(2*\x-1)*(2*\x-1)-5}); %({\x, -2+(2*\x)*(2*\x) }); %({\x,1/(1+\x)});
%\end{tikzpicture}
%\end{center}

Consider the function $g(x) = x \sqrt{4-x^{2}} -1$, part of whose graph is shown below.

\begin{center}
\begin{tikzpicture}[xscale = 4, yscale = 2]
%gray!20,
\fill [white, domain=0:2, variable=\x]
  (0, 0)
  -- plot ({\x}, {\x*sqrt(4-\x*\x)-1})
  -- (2, 0)
  -- cycle;

\draw[<->](-.1, 0) -- (2.3,0);
\draw[<->] (0, -1.1) -- (0,1.1);
\foreach \i in {1,2,3,4,5} {\draw( 1/3*\i, -.1) -- (1/3*\i, .1);}
%\foreach \i in {1,2,3} {\draw( 2/3*\i, -.1) -- (2/3*\i, .1);}
\foreach \i in {-1,1} {\draw(  .05, \i) -- (-.05,\i) node[left]{$\i$};}
\draw  (2/3*3, .1) node[above] {2};
\draw  (1, -.1) node[below] {1};
\draw  (2/3, .4) node[above] {$g(x)$};
%\draw[ultra thick, smooth, domain=0:??4??] plot ({\x, (2*\x-4)**2 - 2});


  \draw[ultra thick, ] plot[smooth, domain=0:2] ({\x}, {\x*sqrt(4-\x*\x)-1}); %({\x, -2+(2*\x)*(2*\x) }); %({\x,1/(1+\x)});
\end{tikzpicture}
\end{center}


\begin{enumerate}
\item Write down, but do not evaluate, a computation that approximates 
%the shaded area 
$\displaystyle{\int_0^2 g(x) \: dx}$ using three {\bf right-hand} rectangles. \fbox{Draw and shade} the rectangles on the graph. (No need to simplify your answer.)\\
\vfill

\item Determine $\displaystyle{\int_0^2 g(x) \: dx}$
%the shaded area 
exactly. Show your work, and simplify your answer.\\
\vfill
\vfill
\vfill
\item How could you make the estimate in (a) more accurate?\\
\vspace{.75in}
%\vfill
\end{enumerate}
%\fbox{Version 2}
%
%Consider the function $g(x) = 1- x \sqrt{4-x^{2}}$, part of whose graph is shown below.
%
%
%\begin{center}
%\begin{tikzpicture}[xscale = 4, yscale = 2]
%\fill [gray!20, domain=0:2, variable=\x]
%  (0, 0)
%  -- plot ({\x}, {-1*\x*sqrt(4-\x*\x)+1})
%  -- (2, 0)
%  -- cycle;
%
%\draw[<->](-.1, 0) -- (2.3,0);
%\draw[<->] (0, -1.1) -- (0,1.1);
%\foreach \i in {1,2,3} {\draw( 2/3*\i, -.1) -- (2/3*\i, .1);}
%\foreach \i in {-1,1} {\draw(  .05, \i) -- (-.05,\i) node[left]{$\i$};}
%\draw  (2/3*3, .1) node[above] {2};
%\draw  (2/3, .4) node[above] {$g(x)$};
%%\draw[ultra thick, smooth, domain=0:??4??] plot ({\x, (2*\x-4)**2 - 2});
%
%
%  \draw[ultra thick, ] plot[smooth, domain=0:2] ({\x}, {1-\x*sqrt(4-\x*\x)}); %({\x, -2+(2*\x)*(2*\x) }); %({\x,1/(1+\x)});
%\end{tikzpicture}
%\end{center}
%\item Write down, but do not evaluate, a computation that approximates the shaded area using three {\bf right-hand} rectangles. \fbox{Draw} the rectangles on the graph.
%
%\begin{enumerate}
%\item Determine the shaded area exactly. Show your work, and simplify your answer.
%
%\item How could you make the estimate in (a) more accurate?
%\end{enumerate}






\newpage

%%%Newoton's method
\item (10 points) Newton's method can be used to find an approximate solution to the equation $x^2 = 8.$ To apply Newton's method to find these roots, let $f(x) = x^2 - 8.$

\begin{enumerate}[(a)]
	\item Use Newton's method with initial approximation $x_0 = 2$ to find $x_1$, a better estimate of a root of the given equation.
	\vfill
	\item Apply one more iteration of Newton's method to find $x_2$.
	\vfill
	\item Notice the equation $x^2 = 8$ has two roots. What value of $x_0$ would make a good choice to find the \textbf{other} root?
	\vfill
\end{enumerate}


\newpage

%\item (E.C.) 
{\bf Extra Credit} (up to 5 points):
\be[(a)]

\item Show that the equation $\cos x - 2x = 4$ has at least one real solution. 

\vfill

\item Show that the equation $\cos x - 2x = 4$ has exactly one real solution. (Hint: what does the derivative tell you about the behavior of the function $g(x) = \cos(x) - 2x - 4$?)

\vfill
\ee


\end{enumerate}
\end{document}
