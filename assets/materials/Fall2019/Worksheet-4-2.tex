\documentclass[11pt,fleqn]{article} 
\usepackage[margin=0.8in, head=0.8in]{geometry} 
\usepackage{amsmath, amssymb, amsthm}
\usepackage{fancyhdr} 
\usepackage{palatino, url, multicol}
\usepackage{graphicx} 
\usepackage[all]{xy}
\usepackage{polynom} 
\usepackage{pdfsync}
\usepackage{enumerate}
\usepackage{framed}
\usepackage{setspace, adjustbox}
\usepackage{array%,tikz, pgfplots
}

\usepackage{tikz, pgfplots}
\usetikzlibrary{calc}
%\pgfplotsset{my style/.append style={axis x line=middle, axis y line=
%middle, xlabel={$x$}, ylabel={$y$}, axis equal }}
%
\pagestyle{fancy} 
\lfoot{UAF Calculus I}
\rfoot{4-2}

\newcommand{\be}{\begin{enumerate}}
\newcommand{\ee}{\end{enumerate}}

\newcommand{\bi}{\begin{itemize}}
\newcommand{\ei}{\end{itemize}}
\def\ds{\displaystyle}
\begin{document}
\setlength{\parindent}{0cm}
\renewcommand{\headrulewidth}{0pt}
\newcommand{\blank}[1]{\rule{#1}{0.75pt}}
\renewcommand{\d}{\displaystyle}
\vspace*{-0.7in}
\begin{center}
 {\large{ \sc{Section 4.2: The Mean Value Theorem}}}
\end{center}

\begin{enumerate}
\item Consider the function $f(x)=x^2$ on the interval $[-1,3]$
	\begin{enumerate}
	\item Find the slope of the secant line of the graph
of $f(x)$ from $x=-1$ to $x=3$.
\vfill
	\item Find a value of $x$ in $[-1,3]$ where $f'(x)$
equals the value in part a.
\vfill
	\item Make a sketch of the graph of $f(x)$ and add to it
the secant line from part a and the tangent line at the location
found in part b.  What property do the secant line and tangent line have?
\vfill
	\end{enumerate}
\item Repeat Problem 1 with the function $g(x) = 1/x$ on $[1,5].$
\vspace{2.5in}
\newpage
\item \textbf{Mean Value Theorem}
\vspace{2.5in}
\item What is the \emph{geometric} meaning of the value $\frac{f(b)-f(a)}{b-a}$?
\vskip 1in
\item Consider the function $f(x)=|x|$ on $[-1,1].$ 
	\begin{enumerate}
	\item What would MVT say about $f$ on $[-1,1]$?
	\vfill
	\item Does MVT ``work" in this case? Why or why not?
	\vfill
	\end{enumerate}
\newpage
\item Suppose $f$ is a continuous function on $[a,b]$ and $f'(x)\ge 0$
for every $x$ in $(a,b)$.  How do $f(a)$ and $f(b)$ compare?
\vfill
\item Suppose $f$ is a continuous function on $[a,b]$ and $f'(x)\le 0$
for every $x$ in $(a,b)$.  How do $f(a)$ and $f(b)$ compare?
\vfill
\item Compare carefully the following two questions, then answer them.
	\begin{enumerate}
	\item Suppose $f(x)=C$ on $[a,b],$ where $C$ is a fixed constant. What can you say about $f'(x)$?
	\vfill
	\item Suppose $f(x)$ is continuous on $[a,b]$  and $f'(x)=0$ on $(a,b).$ What can you say about $f(x)$?
	\vfill
	\end{enumerate}
	\newpage
\item Suppose a car is traveling down the road and in 30 minutes it travels
32.7 miles.  What does the Mean Value Theorem have to say about this?
\vfill
\item Suppose that $f(0) = -3$ and that $f'(x)$ exists and is less than or equal to $ 5$ for all
values of $x$. How large can $f(2)$ possibly be? 
\vfill
\item \textbf{Corollary 7:} If $f'(x)=g'(x)$ for all $x$ in the interval $(a,b)$, then 
$$\framebox(300,50){}.$$

Why?
\vspace{1in}

\end{enumerate}
\end{document}
