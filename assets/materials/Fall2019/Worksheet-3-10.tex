\documentclass[11pt,fleqn]{article} 
\usepackage[margin=0.8in, head=0.8in]{geometry} 
\usepackage{amsmath, amssymb, amsthm}
\usepackage{fancyhdr} 
\usepackage{palatino, url, multicol}
\usepackage{graphicx} 
\usepackage[all]{xy}
\usepackage{polynom} 
\usepackage{pdfsync}
\usepackage{enumerate}
\usepackage{framed}
\usepackage{setspace, adjustbox}
\usepackage{array%,tikz, pgfplots
}

\usepackage{tikz, pgfplots}
\usetikzlibrary{calc}
%\pgfplotsset{my style/.append style={axis x line=middle, axis y line=
%middle, xlabel={$x$}, ylabel={$y$}, axis equal }}
%
\pagestyle{fancy} 
\lfoot{UAF Calculus I}
\rfoot{3-9}

\newcommand{\be}{\begin{enumerate}}
\newcommand{\ee}{\end{enumerate}}

\newcommand{\bi}{\begin{itemize}}
\newcommand{\ei}{\end{itemize}}

\begin{document}
\setlength{\parindent}{0cm}
\renewcommand{\headrulewidth}{0pt}
\newcommand{\blank}[1]{\rule{#1}{0.75pt}}
\renewcommand{\d}{\displaystyle}
\vspace*{-0.7in}
\begin{center}
 {\large{ \sc{Section 3.10: Linearization \& Differentials}}}
\end{center}

\begin{enumerate}
\item Use the linear approximation of $f(x)=\sqrt{x}$ at $x=4$
to approxmiate $\sqrt{4.1}$ and compare your result to its approximation
computed by your calculator.
\vfill

\item  Use the linear approximation to approximate the cosine of $29^\circ=\frac{29}{30}\frac{\pi}{6}$ radians.
\vfill

\item  Find the linear approximation of $f(x)=\ln(x)$ at $a=1$ and use it to approxmate $\ln(0.5)$ and $\ln(0.9)$.  Compare your approximation with your calculator's.
Sketch both the curve $y=\ln(x)$ and $y=L(x)$ and label the points $A=(0.5,\ln(0.5))$ and $B=(0.5,L(0.5))$
\vfill
\newpage
%\aproblem Find the linear approximation of $f(x)=e^x$ at $a=0$ and use 
%it to approximate $e^{0.05}$ and $e^1$  Compare your approximations with your calculator's.
%\vfill
%% \aproblem Find the linear approximation of $f(x)=\sin(\theta)$ at $\theta=0$ and use  it to approximate $\sin(-0.2)$  Compare your approximation with your calculator's.
%\newpage

\item A tree is growing and the radius of its trunk in centemeters
is $r(t)=2\sqrt{t}$ where $t$ is measured in years.  Use the differential
to estimate the change in radius of the tree from 4 years to  
4 years and one month.
\vfill

\item  A coat of paint of thinkness $0.05$cm is being added 
to a hemispherical dome of radius 25m.  Estimate the volume
of paint needed to accomplish this task. [Challenge: will this be an underestimate or an overestimate? Thinking geometrically or thinking algebraically will both give you the same answer.]
\vfill
\item  The radius of a disc is 24cm with an error of $\pm 0.5$cm.
Estimate the error in the area of the disc as an absolute and
as a relative error.
\end{enumerate}
\end{document}
