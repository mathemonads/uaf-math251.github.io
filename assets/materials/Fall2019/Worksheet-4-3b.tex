\documentclass[11pt,fleqn]{article} 
\usepackage[margin=0.8in, head=0.8in]{geometry} 
\usepackage{amsmath, amssymb, amsthm}
\usepackage{fancyhdr} 
\usepackage{palatino, url, multicol}
\usepackage{graphicx, pgfplots} 
\usepackage[all]{xy}
\usepackage{polynom} 
%\usepackage{pdfsync} %% I don't know why this messes up tabular column widths
\usepackage{enumerate}
\usepackage{framed}
\usepackage{setspace}
\usepackage{array}
\usepackage{pgf,tikz}
\usepackage{mathrsfs}
\usetikzlibrary{arrows}

\usetikzlibrary{calc}

\pgfplotsset{compat=1.6}

\pgfplotsset{soldot/.style={color=blue,only marks,mark=*}} \pgfplotsset{holdot/.style={color=blue,fill=white,only marks,mark=*}}

\renewcommand{\headrulewidth}{0pt}
\newcommand{\blank}[1]{\rule{#1}{0.75pt}}
\newcommand{\bc}{\begin{center}}
\newcommand{\ec}{\end{center}}
\newcommand{\be}{\begin{enumerate}}
\newcommand{\ee}{\end{enumerate}}

\newcommand{\ds}{\displaystyle}



\pagestyle{fancy} 
%\lfoot{Uses a calculator}
\rfoot{4-3 Routine Problems}

\begin{document}
\begin{center}
  \large
  \sc{4-3 Derivatives and the Shape of the Graph (Day 2)}\\
\end{center}
\begin{enumerate}
\item Suppose $f(x)=x^5-5x^3.$ 
	\begin{enumerate}
	\item Find all critical points of $f(x).$
	\vspace{1in}
	\item Determine the open  intervals on which $f$ is increasing or decreasing.
	\vspace{1.5in}
	\item Use the First Derivative Test to classify each critical point as a local minimum, a local maximum, or neither.
	\vspace{1.5in}
	\end{enumerate}
\item Draw pictures of graphs that are:
\begin{multicols}{2}
concave up \\

\columnbreak
concave down \\
\end{multicols}
\vfill
\item What can you conclude about the derivatives of the graphs above?
\newpage
\item \textbf{Concavity Test \& Inflection Points}
\vfill
\item Use the Concavity Test to find the intervals of concavity and the inflection points of the function $f(x)=x^5-5x^3.$
\vfill
\item Put the information from problems 1 and 5 together to sketch the shape of the graph.
\vfill
\newpage
\item \textbf{The Second Derivative Test of Local Extrema}
\vspace{2in}

\item Use the Second Derivative Test to classify the only critical point of $f(x)=xe^x.$ Note $f'(x)=(x+1)e^x$ and $f''(x)=(x+2)e^x.$
\vfill
\item Sketch a possible graph of a function $f$ that
satisfies the following conditions:
  \begin{enumerate}
  \item $f$ is continuous and differentiable on $(-\infty,\infty).$
  \item $f(0) = 2$, $f(2) = 3$, $f(4) = 2$
  \item $f'(2) =0$
  \item $f'(x) > 0$ for $ x < 2$ and $f'(x) < 0$ for $2 < x$
     \item $f''(x) > 0$ for $4<x$ and $f''(x) < 0$ for $x <4$. 
  \end{enumerate}
  \vspace{3in}
  \newpage
\item Given the function $f(x) = \ln(x^2 + 4)$ find the
following. 
\begin{enumerate}
\item Determine the domain of $f(x).$\\

    \item Find the intervals of increase or decrease. 
\vfill
  \item Find the local maximum and minimum values. 
\vfill
  \item Find the intervals of concavity and inflection points.
\vfill
  \item Use the information to sketch the graph. 
  \vfill
  \end{enumerate}
\newpage



\end{enumerate}
\end{document}
