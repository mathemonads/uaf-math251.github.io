\documentclass[11pt,fleqn]{article} 
\usepackage[margin=0.8in, head=0.8in]{geometry} 
\usepackage{amsmath, amssymb, amsthm}
\usepackage{fancyhdr} 
\usepackage{palatino, url, multicol}
\usepackage{graphicx} 
\usepackage[all]{xy}
\usepackage{polynom} 
\usepackage{pdfsync}
\usepackage{enumerate}
\usepackage{framed}
\usepackage{setspace, adjustbox}
\usepackage{array%,tikz, pgfplots
}

\usepackage{tikz, pgfplots}
\usetikzlibrary{calc}
%\pgfplotsset{my style/.append style={axis x line=middle, axis y line=
%middle, xlabel={$x$}, ylabel={$y$}, axis equal }}
%
\pagestyle{fancy} 
\lfoot{UAF Calculus I}
\rfoot{5-1}

\newcommand{\be}{\begin{enumerate}}
\newcommand{\ee}{\end{enumerate}}

\newcommand{\bi}{\begin{itemize}}
\newcommand{\ei}{\end{itemize}}
\def\ds{\displaystyle}
\begin{document}
\setlength{\parindent}{0cm}
\renewcommand{\headrulewidth}{0pt}
\newcommand{\blank}[1]{\rule{#1}{0.75pt}}
\renewcommand{\d}{\displaystyle}
\vspace*{-0.7in}
\begin{center}
 {\large\sc{Section 5.1: Distance and Amounts  }}
 \end{center}

\be

\item Suppose the odometer on our car is broken and we
want to estimate the distance driven over a 1.5 hour time period. We take speedometer readings every 15 minutes and then
record them in the table below. Estimate the distance traveled by the
car. What method are you using?

\begin{center}
  \begin{tabular}[ht]{|l|c|c|c|c|c|c|c|}
    \hline 
   Time (minutes) & 0 & 15 &30 &45 &60 & 75 & 90 \\
   \hline
   Velocity (mi/h) &17 & 21 & 24 & 29 & 32 & 31 & 28 \\
  \hline
    \end{tabular}
\end{center}  
\vfill

\item Oil leaked out of a tank at a rate of $r(t)$
liters per hour. The rate decreased as time passed and values of the
rate at 2 hour time intervals are shown in the table. Estimate how much oil leaked out. What method are you using? Is it an overestimate or an underestimate.

\begin{center}
  \begin{tabular}[ht]{|l|c|c|c|c|c|c|}
    \hline
    t (h) & 0 & 2 & 4 & 6 & 8 & 10 \\
   \hline
    r(t) (L/h) & 8.7 & 7.6 & 6.8 & 6.2 & 5.7 & 5.3 \\
   \hline
  \end{tabular}
\end{center}
\vfill


\end{enumerate}

\end{document}
 \end

