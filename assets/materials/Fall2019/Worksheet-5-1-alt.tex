\documentclass[11pt,fleqn]{article} 
\usepackage[margin=0.8in, head=0.8in]{geometry} 
\usepackage{amsmath, amssymb, amsthm}
\usepackage{fancyhdr} 
\usepackage{palatino, url, multicol}
\usepackage{graphicx} 
\usepackage[all]{xy}
\usepackage{polynom} 
\usepackage{pdfsync}
\usepackage{enumerate}
\usepackage{framed}
\usepackage{setspace, adjustbox}
\usepackage{array%,tikz, pgfplots
}

\usepackage{tikz, pgfplots}
\usetikzlibrary{calc}
%\pgfplotsset{my style/.append style={axis x line=middle, axis y line=
%middle, xlabel={$x$}, ylabel={$y$}, axis equal }}
%
\pagestyle{fancy} 
\lfoot{UAF Calculus I}
\rfoot{5-1}

\newcommand{\be}{\begin{enumerate}}
\newcommand{\ee}{\end{enumerate}}

\newcommand{\bi}{\begin{itemize}}
\newcommand{\ei}{\end{itemize}}
\def\ds{\displaystyle}
\begin{document}
\setlength{\parindent}{0cm}
\renewcommand{\headrulewidth}{0pt}
\newcommand{\blank}[1]{\rule{#1}{0.75pt}}
\renewcommand{\d}{\displaystyle}
\vspace*{-0.7in}
\begin{center}
 {\large{ \sc{Section 5.1: Area under a Curve}}}
 \end{center}

\begin{center}
\fbox{{\large {\bf Approximating the area under a curve}}}
\end{center}

Below is a piece of the graph of $y=x^2$. We wish to estimate the area under the curve bounded between the $x$-axis, $x = 0$, $x = 1$, and the curve.

\begin{center}
 \begin{tikzpicture}
\begin{axis}[scale=1,  xtick={.125, .25, ..., 1}, ytick={0,.2, .4, .6, .8, 1},xticklabels={$\frac{1}{8}$, $\frac{1}{4}$, $\frac{3}{8}$, $\frac{1}{2}$, $\frac{5}{8}$, $\frac{3}{4}$, $\frac{7}{8}$, $1$}, 
xmin=0, xmax=1, ymin=0, ymax=1, grid=both, minor y tick num=0,
        minor x tick num=0, mark size=3.0pt,  major grid style={line width=.5pt,draw=black}]
\addplot[domain=0:1,ultra thick]{x*x};
\end{axis}
\end{tikzpicture}
%
\hspace{2cm} 
%
\begin{tikzpicture}[scale = 4]
\draw[fill = gray!50] (0,0) rectangle (.25, .25*.25);
\draw[fill = gray!50] (.25,0) rectangle (.5, .5*.5);
\draw[fill = gray!50] (.5,0) rectangle (.75, .75*.75);
\draw[fill = gray!50] (.75,0) rectangle (1, 1);
\draw[step = .125] (0,0) grid (1,1);
\draw[ultra thick] (0,0) parabola (1,1);
\end{tikzpicture}
\end{center}
%


\be
\item Slice up the axis between $x = 0$ and $x = 1$ into 4 evenly spaced slices. Use the right-hand edge of the slice to construct rectangles whose height is the height of the function at the right-hand edge (i.e., you are using the {\sc right-hand endpoint} of the slice to construct the rectangle). Actually draw the rectangles on the above graph! Your picture should look something like the right hand smaller figure.
%\begin{figure}[ht]
%\begin{center}
%\includegraphics{AreaUnderCurveGraph2}
%\end{center}
%\end{figure}
%
Compute the area of each rectangle: {\bf Make sure to use the height of the rectangle determined from the function:} don't estimate it from the graph! (For example, the height of the first rectangle is $(1/4)^2 = 1/16$)

Area of rectangle 1:
\vfill

Area of rectangle 2:
\vfill

Area of rectangle 3:
\vfill

Area of rectangle 4:
\vfill

Total area of the rectangles: 
\vfill

Is your estimated area bigger or smaller than the total area under the curve? Why?
\vfill

\newpage
\begin{adjustbox}{valign=t,minipage={.45\textwidth}}

\item Now slice up the area under the curve into four equally-spaced slices. Draw in 4 rectangles and use the {\sc left-hand endpoint} of each slice to determine the height of the rectangle. Compute the area of each rectangle.

\end{adjustbox}
\begin{adjustbox}{valign=t,minipage={.45\textwidth}}

\begin{center}
 \begin{tikzpicture}
\begin{axis}[scale=.9,  xtick={.125, .25, ..., 1}, ytick={0,.2, .4, .6, .8, 1},xticklabels={$\frac{1}{8}$, $\frac{1}{4}$, $\frac{3}{8}$, $\frac{1}{2}$, $\frac{5}{8}$, $\frac{3}{4}$, $\frac{7}{8}$, $1$}, 
xmin=0, xmax=1, ymin=0, ymax=1, grid=both, minor y tick num=0,
        minor x tick num=0, mark size=3.0pt,  major grid style={line width=.5pt,draw=black}]
\addplot[domain=0:1,ultra thick]{x*x};
\end{axis}
\end{tikzpicture}
\end{center}
%\begin{figure}[ht]
%\begin{center}
%\includegraphics{AreaUnderCurveGraph}
%\end{center}
%\end{figure}
\end{adjustbox}

Determine the total area of the rectangles (use the function to determine the height, not the graph!):

\vspace{1in}

Is this an overestimation or an underestimation? Why?
\vfill

%\newpage
\begin{adjustbox}{valign=t,minipage={.45\textwidth}}

\item Now slice up the segment between $x = 0$ and $x = 1$ into four equal pieces. Using the {\sc midpoint} of each piece to determine the height of each rectangle, draw in four rectangles, and determine the area of each rectangle (use the function to determine the heights, not the picture!).
\end{adjustbox}
%\begin{figure}[ht]
%\begin{center}
%\includegraphics{AreaUnderCurveGraph}
%\end{center}
%\end{figure}
\begin{adjustbox}{valign=t,minipage={.45\textwidth}}
\begin{center}
 \begin{tikzpicture}
\begin{axis}[scale=.9,  xtick={.125, .25, ..., 1}, ytick={0,.2, .4, .6, .8, 1},xticklabels={$\frac{1}{8}$, $\frac{1}{4}$, $\frac{3}{8}$, $\frac{1}{2}$, $\frac{5}{8}$, $\frac{3}{4}$, $\frac{7}{8}$, $1$}, 
xmin=0, xmax=1, ymin=0, ymax=1, grid=both, minor y tick num=0,
        minor x tick num=0, mark size=3.0pt,  major grid style={line width=.5pt,draw=black}]
\addplot[domain=0:1,ultra thick]{x*x};
\end{axis}
\end{tikzpicture}
\end{center}
\end{adjustbox}

Determine the total area of the rectangles (use the function to determine the heights, not the picture!):

\vspace{.75in}

What do you think: based on your rectangles, is this an overestimation or an underestimation? (The other two rectangle types should have been obvious; this one is considerably more subtle. Make some sort of argument in favor of one choice or the other.)
%\vfill
\vspace{.75in}

\newpage

\begin{center}
\fbox{{\bf Better area approximations}}
\end{center}

\begin{adjustbox}{valign=t,minipage={.45\textwidth}}

\item Now we want to try to get a better estimation for the area. This time draw in 8 rectangles, using the {\sc right-hand endpoint} to construct each rectangle.

\end{adjustbox}
\begin{adjustbox}{valign=t,minipage={.45\textwidth}}
\begin{center}
 \begin{tikzpicture}
\begin{axis}[scale=.8,  xtick={.125, .25, ..., 1}, ytick={0,.2, .4, .6, .8, 1},xticklabels={$\frac{1}{8}$, $\frac{1}{4}$, $\frac{3}{8}$, $\frac{1}{2}$, $\frac{5}{8}$, $\frac{3}{4}$, $\frac{7}{8}$, $1$}, 
xmin=0, xmax=1, ymin=0, ymax=1, grid=both, minor y tick num=0,
        minor x tick num=0, mark size=3.0pt,  major grid style={line width=.5pt,draw=black}]
\addplot[domain=0:1,ultra thick]{x*x};
\end{axis}
\end{tikzpicture}
\end{center}
\end{adjustbox}

Write down a calculation (you don't have to actually do the computation) to determine the area of the 8 rectangles. Is this area a more accurate estimation of the area under the curve than when you used 4 rectangles? Why?
\vfill

%\item Suppose you had $n$ equally-spaced rectangles, labelled $1, 2, \ldots, i, \ldots, n$, constructed between $x = 0$ and $x = 1$ (like the 8 above, only this time with $n$).  If they were constructed using the right-hand endpoint, what is the height of rectangle $i$? What is the width?
%\bigskip
%
%width of rectangle $i$ \hrulefill
%
%
%
%\bigskip
%
%height of rectangle $i$ \hrulefill
%
%\bigskip
%
%
%area of rectangle $i$ \hrulefill
%
%\bigskip
%
%Write down a computation giving the total area of the $n$ rectangles, and simplify it as much as possible. (Since you don't know what $n$ is, you can't write all $n$ terms, so use $\ldots$ to indicate the ones you don't write down.)
%\vfill
%\vfill
%
%\newpage
%\item It is a fact that \(1^2 + 2^2 + 3^2 + \ldots + (n-1)^2 + n^2 = \frac{n(n+1)(2n+1)}{6}.\) Use this fact to rewrite the total area of $n$ rectangles. Simplify your answer.
%
%\vfill
%
%\vfill
%\item Write down a statement involving limits which uses right-hand rectangles and gives ``the best'' approximation of the area under the curve (you should use what you have previously figured out!) (You will get better and better approximations as $n$ gets bigger and bigger (so you have more, skinnier, rectangles). How can you translate this into a statement about limits?)
%\vfill
%
%\item Compute the limit. We are actually going to define the area under the curve to {\bf be} that limit, so the value you have computed \emph{is} the area under the curve. Of the three approximations of the area under the curve using 4 rectangles (left-hand endpoint, right-hand endpoint, midpoint), which gave the best approximation?
%\vfill 
%\vfill
%\vfill
%

\item Suppose the odometer on our car is broken and we
want to estimate the distance driven over a 1.5 hour time period. We take speedometer readings every 15 minutes and then
record them in the table below. Estimate the distance traveled by the
car. What method are you using?

\begin{center}
  \begin{tabular}[ht]{|l|c|c|c|c|c|c|c|}
    \hline 
   Time (minutes) & 0 & 15 &30 &45 &60 & 75 & 90 \\
   \hline
   Velocity (mi/h) &17 & 21 & 24 & 29 & 32 & 31 & 28 \\
  \hline
    \end{tabular}
\end{center}  
\vfill

\item Oil leaked out of a tank at a rate of $r(t)$
liters per hour. The rate decreased as time passed and values of the
rate at 2 hour time intervals are shown in the table. Estimate how much oil leaked out. What method are you using? Is it an overestimate or an underestimate.

\begin{center}
  \begin{tabular}[ht]{|l|c|c|c|c|c|c|}
    \hline
    t (h) & 0 & 2 & 4 & 6 & 8 & 10 \\
   \hline
    r(t) (L/h) & 8.7 & 7.6 & 6.8 & 6.2 & 5.7 & 5.3 \\
   \hline
  \end{tabular}
\end{center}
\vfill


\end{enumerate}

\end{document}
 \end

