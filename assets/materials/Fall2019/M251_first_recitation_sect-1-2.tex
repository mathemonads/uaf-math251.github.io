\documentclass[12pt]{article}

\usepackage{graphicx,color,enumerate,multicol}
\usepackage[top=1in,bottom=0.5in,left=1in, right=1in]{geometry}

\usepackage{tikz}

%% Use Minion fonts if available.  Otherwise Times.
\IfFileExists{MinionPro.sty}{\usepackage[lf]{MinionPro}}{}
\usepackage{amsmath,amsthm,amsbsy}
\IfFileExists{MinionPro.sty}{}{\usepackage{times,txfonts}}

%% Setup aproblem environment, 
%% aproblem items
%% subproblems environment
%% subproblem items
\makeatletter
\newcounter{probcount}
\newcounter{subprobcount}
\newlength\probsep
\newlength\pshrinking
\newif\iffirstprob
\newenvironment{aproblems}%
  {\ifhmode\unskip\par\fi\setcounter{probcount}{0}\probsep\parskip
  \sbox\@tempboxa{\textbf{9.}}\pshrinking\wd\@tempboxa\advance\pshrinking\labelsep
  \let\hproblem\aproblem
  \advance\linewidth -\pshrinking
  \advance\@totalleftmargin\pshrinking
  \advance\leftskip\pshrinking}%
  {\ifhmode\unskip \par\fi\advance\leftskip-\pshrinking}%

\newcommand{\aproblem}{%
  \setcounter{subprobcount}{0}%
  \stepcounter{probcount}%
  \def\@currentlabel{\arabic{probcount}}%
  \ifhmode
    \unskip \par
  \fi
%  \addpenalty{-4000}%
  \iffirstprob\else\addvspace\probsep\fi
  \firstprobfalse
  \hskip -\labelwidth\hskip -\labelsep 
  \hbox to\labelwidth{\hss\textbf{\arabic{probcount}.}}\hskip\labelsep
}%

\newcommand{\subprob}{\item\def\@currentlabel{\arabic{probcount}\alph{\thelistlabel}}}
\newcommand{\skipproblem}{\stepcounter{probcount}}


%% The following commands put defined left and right headers on the top, and a page number
%% on the bottom of all pages beyond page 1
\usepackage{fancyhdr}
\pagestyle{fancy}
\fancyfoot[C]{\ifnum \value{page} > 1\relax\thepage\fi}
\fancyhead[L]{\ifx\@doclabel\@empty\else\@doclabel\fi}
\fancyhead[R]{\ifx\@docdate\@empty\else\@docdate\fi}
\headheight 15pt
\def\doclabel#1{\gdef\@doclabel{#1}}
\def\docdate#1{\gdef\@docdate{#1}}
\makeatother

%% General formatting parameters
\parindent 0pt
\parskip 6pt plus 1pt


\doclabel{Math F251: Section 1.2 Worksheet}
\docdate{27 August 2019}

\begin{document}
\renewcommand{\d}{\displaystyle}
\aproblem Find an equation of a line that has slope $-2$
and passes through the point $(3,-5)$.  Write the equation
of the line in point-slope form \textbf{and} then again 
in $y$-intercept form.
\vfill

\aproblem Suppose the average surface temperature of the earth
is modeled by the linear function
\[
T = 0.02t + 8.50
\]
where $T$ is temperature in C$^{\circ}$ and $t$ represents years since 1900. 

\begin{quote}
  \begin{enumerate}[(a)]
  \item What units do the slope and the $T$-intercept have?
  \vskip 1cm
  \item What do the slope and $T$-intercept represent in physical terms?  \vskip 2cm
  \item Use the equation to predict the average global surface
    temperature in 2100.  \vskip1cm
    \item Rewrite the formula for temperature above 
in point-slope form where the point is determined by 
the the temperature in the year 2100.
\vfill

  \end{enumerate}
\end{quote}

\newpage
\subsection*{Quadratic Functions}

\aproblem A ball is dropped from the upper observation deck
of the CN Tower 450 m above the ground. The height above the ground
$h$ after $t$ seconds is given by the equation $h(t) = - 4.9 t^2 +
0.96 t + 449.36$.
\begin{quote}
  \begin{enumerate}[(a)]
  \item When does the ball hit the ground?  \vfill

  \item Sketch a rough picture of the graph $h(t)$. Given the
  physical understanding of the problem, what would be
  a reasonable domain for the function $h(t)$?
  \vfill

  \end{enumerate}
\end{quote}

\subsection*{Essential Graphs}
Set your calculator/computer aside. You should know the 
graphs the following functions in this section by 
heart. In your sketches, clearly indicate any asymptotes and intercepts.

\aproblem Sketch the graphs of the following functions:


  \begin{multicols}{3}{
      % make sure you added \usepackage{enumerate}
      \vspace*{-0.5in}
      \begin{enumerate}[(a)]
      \item $y = x$

\begin{tikzpicture}[scale=0.5][>=latex]
%x axis
\draw[->] (-4 ,0) -- (4 ,0) node[below] {$x$};
\foreach \x in {-3,...,3}
\draw[shift={(\x,0)}] (0pt,2pt) -- (0pt,-2pt);
%y axis
\draw[->] (0,-4) -- (0,4) node[left] {$y$};
\foreach \y in {-3,...,3}
\draw[shift={(0,\y)}] (2pt,0pt) -- (-2pt,0pt);
% \node[below left] at (0,0) {\footnotesize $0$};
\end{tikzpicture}

      \item $y = x^2$

\begin{tikzpicture}[scale=0.5][>=latex]
%x axis
\draw[->] (-4 ,0) -- (4 ,0) node[below] {$x$};
\foreach \x in {-3,...,3}
\draw[shift={(\x,0)}] (0pt,2pt) -- (0pt,-2pt);
%y axis
\draw[->] (0,-4) -- (0,4) node[left] {$y$};
\foreach \y in {-3,...,3}
\draw[shift={(0,\y)}] (2pt,0pt) -- (-2pt,0pt);
% \node[below left] at (0,0) {\footnotesize $0$};
\end{tikzpicture}

      \item $y = x^3$

\begin{tikzpicture}[scale=0.5][>=latex]
%x axis
\draw[->] (-4 ,0) -- (4 ,0) node[below] {$x$};
\foreach \x in {-3,...,3}
\draw[shift={(\x,0)}] (0pt,2pt) -- (0pt,-2pt);
%y axis
\draw[->] (0,-4) -- (0,4) node[left] {$y$};
\foreach \y in {-3,...,3}
\draw[shift={(0,\y)}] (2pt,0pt) -- (-2pt,0pt);
% \node[below left] at (0,0) {\footnotesize $0$};
\end{tikzpicture}

      \end{enumerate}}
  \end{multicols}

\newpage

\aproblem Sketch the graphs of the following functions:


  \begin{multicols}{3}{
      % make sure you added \usepackage{enumerate}
      \vspace*{-0.5in}
      \begin{enumerate}[(a)]
      \item $y = \sqrt{x}$ 

\begin{tikzpicture}[scale=0.5][>=latex]
%x axis
\draw[->] (-4 ,0) -- (4 ,0) node[below] {$x$};
\foreach \x in {-3,...,3}
\draw[shift={(\x,0)}] (0pt,2pt) -- (0pt,-2pt);
%y axis
\draw[->] (0,-4) -- (0,4) node[left] {$y$};
\foreach \y in {-3,...,3}
\draw[shift={(0,\y)}] (2pt,0pt) -- (-2pt,0pt);
% \node[below left] at (0,0) {\footnotesize $0$};
\end{tikzpicture}

      \item $y = \sqrt[3]{x}$

\begin{tikzpicture}[scale=0.5][>=latex]
%x axis
\draw[->] (-4 ,0) -- (4 ,0) node[below] {$x$};
\foreach \x in {-3,...,3}
\draw[shift={(\x,0)}] (0pt,2pt) -- (0pt,-2pt);
%y axis
\draw[->] (0,-4) -- (0,4) node[left] {$y$};
\foreach \y in {-3,...,3}
\draw[shift={(0,\y)}] (2pt,0pt) -- (-2pt,0pt);
% \node[below left] at (0,0) {\footnotesize $0$};
\end{tikzpicture}

      \item $y = \d \frac 1 x$

\begin{tikzpicture}[scale=0.5][>=latex]
%x axis
\draw[->] (-4 ,0) -- (4 ,0) node[below] {$x$};
\foreach \x in {-3,...,3}
\draw[shift={(\x,0)}] (0pt,2pt) -- (0pt,-2pt);
%y axis
\draw[->] (0,-4) -- (0,4) node[left] {$y$};
\foreach \y in {-3,...,3}
\draw[shift={(0,\y)}] (2pt,0pt) -- (-2pt,0pt);
% \node[below left] at (0,0) {\footnotesize $0$};
\end{tikzpicture}

      \end{enumerate}}
  \end{multicols}


\aproblem Sketch the graphs of the following functions:


  \begin{multicols}{3}{
      % make sure you added \usepackage{enumerate}
      \vspace*{-0.5in}
      \begin{enumerate}[(a)]
      \item $y = \d \frac{1}{x^2}$

\begin{tikzpicture}[scale=0.5][>=latex]
%x axis
\draw[->] (-4 ,0) -- (4 ,0) node[below] {$x$};
\foreach \x in {-3,...,3}
\draw[shift={(\x,0)}] (0pt,2pt) -- (0pt,-2pt);
%y axis
\draw[->] (0,-4) -- (0,4) node[left] {$y$};
\foreach \y in {-3,...,3}
\draw[shift={(0,\y)}] (2pt,0pt) -- (-2pt,0pt);
% \node[below left] at (0,0) {\footnotesize $0$};
\end{tikzpicture}

      \item $y = e^x$

\begin{tikzpicture}[scale=0.5][>=latex]
%x axis
\draw[->] (-4 ,0) -- (4 ,0) node[below] {$x$};
\foreach \x in {-3,...,3}
\draw[shift={(\x,0)}] (0pt,2pt) -- (0pt,-2pt);
%y axis
\draw[->] (0,-4) -- (0,4) node[left] {$y$};
\foreach \y in {-3,...,3}
\draw[shift={(0,\y)}] (2pt,0pt) -- (-2pt,0pt);
% \node[below left] at (0,0) {\footnotesize $0$};
\end{tikzpicture}


      \item $y = \ln x$

\begin{tikzpicture}[scale=0.5][>=latex]
%x axis
\draw[->] (-4 ,0) -- (4 ,0) node[below] {$x$};
\foreach \x in {-3,...,3}
\draw[shift={(\x,0)}] (0pt,2pt) -- (0pt,-2pt);
%y axis
\draw[->] (0,-4) -- (0,4) node[left] {$y$};
\foreach \y in {-3,...,3}
\draw[shift={(0,\y)}] (2pt,0pt) -- (-2pt,0pt);
% \node[below left] at (0,0) {\footnotesize $0$};
\end{tikzpicture}
      \end{enumerate}}
  \end{multicols}


\aproblem Sketch the following functions on $[- 2 \pi, 2
\pi]$

%\begin{quote}
  \begin{multicols}{2}{
      % make sure you added \usepackage{enumerate}
      \vspace*{-0.45in}
      \begin{enumerate}[(a)]
      \item $y = \sin x$

\begin{tikzpicture}[scale=0.65][>=latex]
%x axis
\draw[->] (-5 ,0) -- (5 ,0) node[below] {$x$};
\foreach \x in {-4,...,4}
\draw[shift={(\x,0)}] (0pt,2pt) -- (0pt,-2pt);
%y axis
\draw[->] (0,-2.5) -- (0,2.5) node[left] {$y$};
\foreach \y in {-1,...,1}
\draw[shift={(0,\y)}] (2pt,0pt) -- (-2pt,0pt);
% \node[below left] at (0,0) {\footnotesize $0$};
\end{tikzpicture}
\item $y = \cos x$

\begin{tikzpicture}[scale=0.65][>=latex]
%x axis
\draw[->] (-5 ,0) -- (5 ,0) node[below] {$x$};
\foreach \x in {-4,...,4}
\draw[shift={(\x,0)}] (0pt,2pt) -- (0pt,-2pt);
%y axis
\draw[->] (0,-2.5) -- (0,2.5) node[left] {$y$};
\foreach \y in {-1,...,1}
\draw[shift={(0,\y)}] (2pt,0pt) -- (-2pt,0pt);
% \node[below left] at (0,0) {\footnotesize $0$};
\end{tikzpicture}
\columnbreak
\item $y = \tan x$

\begin{tikzpicture}[scale=0.65][>=latex]
%x axis
\draw[->] (-5 ,0) -- (5 ,0) node[below] {$x$};
\foreach \x in {-4,...,4}
\draw[shift={(\x,0)}] (0pt,2pt) -- (0pt,-2pt);
%y axis
\draw[->] (0,-2.5) -- (0,2.5) node[left] {$y$};
\foreach \y in {-1,...,1}
\draw[shift={(0,\y)}] (2pt,0pt) -- (-2pt,0pt);
% \node[below left] at (0,0) {\footnotesize $0$};
\end{tikzpicture}

\end{enumerate}}
\end{multicols}
%\end{quote}
\newpage
\subsection*{Graphs of Functions Related by Transformations}

\vskip -0.4cm
Once you know the graph of one function, it's easy to sketch
the graphs of other functions that related to it by 
certain simple transformations. Again, set technology aside
and sketch these the old-fashioned way.

\aproblem  \textbf{Translations}. Graph the following functions.

\begin{quote}
  \begin{multicols}{2}{
      % make sure you added \usepackage{enumerate}
      \vspace*{-0.45in}
      \begin{enumerate}[a)]
      \item $y = x^2 - 2$

\begin{tikzpicture}[scale=0.5][>=latex]
%x axis
\draw[->] (-5 ,0) -- (5 ,0) node[below] {$x$};
\foreach \x in {-4,...,4}
\draw[shift={(\x,0)}] (0pt,2pt) -- (0pt,-2pt);
%y axis
\draw[->] (0,-5) -- (0,5) node[left] {$y$};
\foreach \y in {-4,...,4}
\draw[shift={(0,\y)}] (2pt,0pt) -- (-2pt,0pt);
% \node[below left] at (0,0) {\footnotesize $0$};
\end{tikzpicture}

      \item $y = (x+2)^2$

\begin{tikzpicture}[scale=0.5][>=latex]
%x axis
\draw[->] (-5 ,0) -- (5 ,0) node[below] {$x$};
\foreach \x in {-4,...,4}
\draw[shift={(\x,0)}] (0pt,2pt) -- (0pt,-2pt);
%y axis
\draw[->] (0,-5) -- (0,5) node[left] {$y$};
\foreach \y in {-4,...,4}
\draw[shift={(0,\y)}] (2pt,0pt) -- (-2pt,0pt);
% \node[below left] at (0,0) {\footnotesize $0$};
\end{tikzpicture}
      \end{enumerate}}
  \end{multicols}
\end{quote}

\aproblem \textbf{Refections.} Sketch the graphs of the  following functions.

\vskip -0.4cm
\begin{quote}
  \begin{multicols}{2}{
      % make sure you added \usepackage{enumerate}
      \vspace*{-0.45in}
      \begin{enumerate}[a)]
      \item $y = - \sqrt{x}$ 

\begin{tikzpicture}[scale=0.5][>=latex]
%x axis
\draw[->] (-5 ,0) -- (5 ,0) node[below] {$x$};
\foreach \x in {-4,...,4}
\draw[shift={(\x,0)}] (0pt,2pt) -- (0pt,-2pt);
%y axis
\draw[->] (0,-5) -- (0,5) node[left] {$y$};
\foreach \y in {-4,...,4}
\draw[shift={(0,\y)}] (2pt,0pt) -- (-2pt,0pt);
% \node[below left] at (0,0) {\footnotesize $0$};
\end{tikzpicture}

      \item $y = \sqrt{-x}$

\begin{tikzpicture}[scale=0.5][>=latex]
%x axis
\draw[->] (-5 ,0) -- (5 ,0) node[below] {$x$};
\foreach \x in {-4,...,4}
\draw[shift={(\x,0)}] (0pt,2pt) -- (0pt,-2pt);
%y axis
\draw[->] (0,-5) -- (0,5) node[left] {$y$};
\foreach \y in {-4,...,4}
\draw[shift={(0,\y)}] (2pt,0pt) -- (-2pt,0pt);
% \node[below left] at (0,0) {\footnotesize $0$};
\end{tikzpicture}
      \end{enumerate}}
  \end{multicols}
\end{quote}

\vskip -0.4cm
\aproblem  Graph the following functions.
\begin{multicols}{3}{
      % make sure you added \usepackage{enumerate}
      \vspace*{-0.45in}
      \begin{enumerate}[(a)]
      \item $f(x) = 4 - x^2$

\begin{tikzpicture}[scale=0.45][>=latex]
%x axis
\draw[->] (-5 ,0) -- (5 ,0) node[below] {$x$};
\foreach \x in {-4,...,4}
\draw[shift={(\x,0)}] (0pt,2pt) -- (0pt,-2pt);
%y axis
\draw[->] (0,-5) -- (0,5) node[left] {$y$};
\foreach \y in {-4,...,4}
\draw[shift={(0,\y)}] (2pt,0pt) -- (-2pt,0pt);
% \node[below left] at (0,0) {\footnotesize $0$};
\end{tikzpicture}
 
 \item $f(x) = e^{-x} - 3$ 

\begin{tikzpicture}[scale=0.45][>=latex]
%x axis
\draw[->] (-5 ,0) -- (5 ,0) node[below] {$x$};
\foreach \x in {-4,...,4}
\draw[shift={(\x,0)}] (0pt,2pt) -- (0pt,-2pt);
%y axis
\draw[->] (0,-5) -- (0,5) node[left] {$y$};
\foreach \y in {-4,...,4}
\draw[shift={(0,\y)}] (2pt,0pt) -- (-2pt,0pt);
% \node[below left] at (0,0) {\footnotesize $0$};
\end{tikzpicture}

\item $f(x) = \ln (x + 3) + 1$ 

\begin{tikzpicture}[scale=0.45][>=latex]
%x axis
\draw[->] (-5 ,0) -- (5 ,0) node[below] {$x$};
\foreach \x in {-4,...,4}
\draw[shift={(\x,0)}] (0pt,2pt) -- (0pt,-2pt);
%y axis
\draw[->] (0,-5) -- (0,5) node[left] {$y$};
\foreach \y in {-4,...,4}
\draw[shift={(0,\y)}] (2pt,0pt) -- (-2pt,0pt);
% \node[below left] at (0,0) {\footnotesize $0$};
\end{tikzpicture}
      \end{enumerate}}
\end{multicols}
%\aproblem (bonus question) Explain the difference between a \emph{linear} function and a \emph{nonlinear} function. Give both a graphical explanation and an algebraic explanation.
%\vfill
\end{document}