\documentclass[11pt,fleqn]{article} 
\usepackage[margin=0.8in, head=0.8in]{geometry} 
\usepackage{amsmath, amssymb, amsthm}
\usepackage{fancyhdr} 
\usepackage{palatino, url, multicol}
\usepackage{graphicx,latexsym} 
\usepackage[all]{xy}
\usepackage{polynom} 
\usepackage{pdfsync}
\usepackage{enumerate, enumitem}
\usepackage{framed}
\usepackage{setspace}
\usepackage{array,tikz,xcolor, pgfplots}
\pagestyle{fancy} 
\lfoot{UAF Calculus 1}
\rfoot{\S 1.4 \& 1.5 }

\usetikzlibrary{calc}
\pgfplotsset{my style/.append style={axis x line=middle, axis y line=
middle, xlabel={$x$}, ylabel={$y$}, axis equal }}

\begin{document}
\renewcommand{\headrulewidth}{0pt}
\newcommand{\blank}[1]{\rule{#1}{0.75pt}}
\renewcommand{\d}{\displaystyle}


\vspace*{-1in}
\begin{center}
   \sc{Lecture Notes: \S 2.1 }
\end{center}
\begin{enumerate}
\item The point $P(2,3)$ lies on the graph of $f(x)=x+\frac{2}{x}.$
	\begin{enumerate}
	\item If possible, find the slope of the secant line between the point $P$ and each of the points with $x$ values listed below. For each estimate the slope to 4 decimal places. NOTE: You do not need the graph of the function to answer this numerical question.\\
		{\LARGE{\begin{center}
		\begin{tabular}{l | l | c}
		\multicolumn{2}{c}{point $Q$}& slope of secant line $PQ$\\
		$x$-value&\quad$y$-value \quad \quad& $PQ$\\
		\hline
		$x=4$&&\\
		\hline
		$x=3$&&\\
		\hline
		$x=2.5$&&\\
		\hline
		$x=2.25$&&\\
		\hline
		$x=2.1$&&\\
		\hline
		$x=0$&&\\
		\hline
		$x=1$&&\\
		\hline
		$x=1.5$&&\\
		\hline
		$x=1.75$&&\\
		\hline
		$x=1.9$&&\\
		\hline
		\end{tabular}
		\end{center}}}
	\item  Now, use technology to sketch a rough graph $f(x)$ on the interval $(0,5]$ and add the secant lines from part $a$. (Your graph may be messy...It's ok.) Label the secant lines with their respective slopes. What can you conclude about the slope of the tangent line to $f(x)$ at $x=2$?
	\vfill
	\item Write a best guess for the equation of the line tangent to $f(x)$ at point $P$. Is your equation plausible?
	\vspace{.5in}
	\end{enumerate}
	\newpage

\item The table shows the position of a cyclist after accelerating from rest.\\

\begin{tabular}{|c||c|c|c|c|c|c|c|c|c|}
$t$ (minutes) &0&30&60&90&120&150&180&210&240\\
\hline
$d$ (miles) &0&9.2&18.7&23.1&38.1&46.6&59.7&72.6&80\\
\end{tabular}
\begin{enumerate}
\item Estimate the cyclist's average velocity in miles per hour  during:
\begin{enumerate}
\item the first hour\\ \vfill
\item the second hour\\ \vfill
\item the third hour\\ \vfill
\item the fourth hour\\ \vfill
\end{enumerate}
\item Estimate the cyclist's average velocity (in miles per hour) in the time period $[60,90]$.\\  \vfill
\item Estimate the cyclist's average velocity (in miles per hour) in the time period $[90,120]$.\\  \vfill
\item Estimate how fast the cyclist was going 1.5 hours into the ride.\\  \vfill
\item During what period do you estimate the cyclist was riding the fastest on average?\\ \vfill
\item What does any this have to do with secant lines and tangent lines?
\end{enumerate}
%
%	\end{enumerate}
%	\end{multicols}
%	\vfill
%\item Evaluate $\sin^{-1}(1).$
%\vfill
%\item Find the exact value of each expression.
%\begin{multicols}{2}
%\begin{enumerate}
%	\item $\log_2 16$
%	\item $e^{\ln 5}$
%	\end{enumerate}
%	\end{multicols}
%	\vfill
%
%\newpage
%\item Solve each equation below for $x$.
%\begin{multicols}{2}
%\begin{enumerate}
%	\item $10=2e^{x+1}$
%	\item $\ln (x^2-1)=1$
%	\end{enumerate}
%	\end{multicols}
%	\vfill
%\item Sketch each function. Include domain, range, intercepts and asymptotes.\\
%\begin{multicols}{2}
%\begin{enumerate}
%	\item $f(x)=\ln(x+1)$
%	
%	\tikz{ \draw[<->] (-4,0) -- (4,0); \draw [<->] (0,-4) -- (0,4);}
%	
%	\item $f(x)=- \ln x$
%	
%	\tikz{ \draw[<->] (-4,0) -- (4,0); \draw [<->] (0,-4) -- (0,4);}
%	
%	\end{enumerate}
%	\end{multicols}
%	\vfill
%
\end{enumerate}
\end{document}