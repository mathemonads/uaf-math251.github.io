\documentclass[11pt,fleqn]{article} 
\usepackage[margin=0.8in, head=0.8in]{geometry} 
\usepackage{amsmath, amssymb, amsthm}
\usepackage{fancyhdr} 
\usepackage{palatino, url, multicol}
\usepackage{graphicx,latexsym} 
\usepackage[all]{xy}
\usepackage{polynom} 
\usepackage{pdfsync}
\usepackage{enumerate, enumitem}
\usepackage{framed}
\usepackage{setspace}
\usepackage{array,tikz,xcolor, pgfplots}
\pagestyle{fancy} 
\lfoot{UAF Calculus 1}
\rfoot{\S 1.4 \& 1.5 }

\usetikzlibrary{calc}
\pgfplotsset{my style/.append style={axis x line=middle, axis y line=
middle, xlabel={$x$}, ylabel={$y$}, axis equal }}

\begin{document}
\renewcommand{\headrulewidth}{0pt}
\newcommand{\blank}[1]{\rule{#1}{0.75pt}}
\renewcommand{\d}{\displaystyle}


\vspace*{-0.7in}
\begin{center}
  \LARGE \sc{Lecture Notes: \S 1.4 }
\end{center}
\begin{enumerate}
\item Use the Laws of Exponents to rewrite and simplify. Write down the rules that you are using to the side of your work.
\begin{multicols}{2}
\begin{enumerate}
\item $(25^2)(5^{-3})$
\item $\sqrt[3]{x^{-2}}$
\end{enumerate} 
\end{multicols}
\vfill
\begin{multicols}{2}
\begin{enumerate}[label=\alph*.,start=3]
\item $b^{(n-1)}(3b^2)^n$
\item $\frac{6x^2y}{\sqrt{4xy^3}}$
\end{enumerate} 
\end{multicols}
\vfill
\item On the axes below, graph $f(x)=2^x,\: g(x)=e^x,\: h(x)=10^x,$ and $k(x)=\left(\frac{1}{2} \right)^x.$ Label any $x$- and $y$-intercepts.

\begin{center}
\tikz{ \draw[<->] (-7,0) -- (7,0); \draw [<->] (0,-2) -- (0,4);}\end{center}
%\vspace{3in}
%\item Are the two functions $f(x)=\left( \frac{1}{2} \right)^{x}$ and $g(x)=2^{-x}$ the same function or not? Justify your answer. (BONUS: Can you give both an algebraic justification and a geometric justification.)
%\vfill
\newpage
\item Assume $a >0.$ What is the domain and range of $f(x)=a^x$? Asymptotes?
\vspace{1in}

\item Sketch the graph of each function below, using what you know about transformations of functions. Determine its domain and range, and label any $x$- and $y$-intercepts (use exact numbers) and horizontal or vertical asymptotes.
\begin{enumerate}
\begin{multicols}{2}
\item $f(x)=1-2^{x}$
\columnbreak
\item $y=2e^{x-2}$
\end{multicols}
\end{enumerate}
\vfill
%\item Without the use of a calculator, compute the following:
%\begin{enumerate}
%\item $ \log_2 \frac{1}{16}=$\\
%
%\item $\ln e^{0.24}=$ \\
%
%\item $e^{5 \ln x}=$\\
%
%
%\end{enumerate}
%
%\item On the same set of axes, graph $f(x)=e^x$ and $g(x)=\ln x.$
%\vfill
%\newpage
%\item Solve the following equations for
%  $x$.
%
%  \begin{multicols}{2}{
%      % make sure you added \usepackage{enumerate}
%      \vspace*{-0.35in}
%      \begin{enumerate}[label=\alph*.]
%      \item $ \ln (x+5) - 1 = 7$
%      \item  $e^{2x-5} + 4 = 10$
%      \end{enumerate}}
%  \end{multicols}
%\vfill

\item Are the following statements true or false? If
either case, explain why.  If possible, change the false statements so
that they are a true statement. 


  \begin{enumerate}[label=\alph*.]
  \item $(a+b)^2 = a^2 + b^2$ \vskip0.25in
  \item $\sqrt{x^2 + 4} = x + 2$ \vskip0.25in
  \item $\d \frac{a+b}{c+d} = \frac a c + \frac b d$ \vskip0.25in
  \item $\d \frac{a+b}{c} = \frac a c + \frac b c$ \vskip0.25in
%  \item $\ln(x + y) = \ln x + \ln y$ \vskip0.25in
%  \item $\d \frac{ \ln x}{\ln y} = \ln \left( \frac x y \right)$
%    \vskip0.25in
%  \item $\d \ln(x - y) = \ln \left( \frac x y \right)$ \vskip0.25in
%  \item $f^{-1} (x) = \frac 1 {f(x)}$ \vskip0.25in
%  \item $f^2(x) = (f(x))^2$ \vskip0.25in
  \end{enumerate}


\end{enumerate}
\end{document}