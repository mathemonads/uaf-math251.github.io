\documentclass[11pt,fleqn]{article} 
\usepackage[margin=0.8in, head=0.8in]{geometry} 
\usepackage{amsmath, amssymb, amsthm}
\usepackage{fancyhdr} 
\usepackage{palatino, url, multicol, tabularx}
\usepackage{graphicx} 
\usepackage[all]{xy}
\usepackage{polynom} 
\usepackage{pdfsync}
\usepackage{enumerate}
\usepackage{framed}
\usepackage{setspace, adjustbox}
\usepackage{array%,tikz, pgfplots
}

\usepackage{tikz, pgfplots}
\usetikzlibrary{calc}
%\pgfplotsset{my style/.append style={axis x line=middle, axis y line=
%middle, xlabel={$x$}, ylabel={$y$}, axis equal }}
%
\pagestyle{fancy} 
\lfoot{UAF Calculus I}
\rfoot{5-2}

\newcommand{\be}{\begin{enumerate}}
\newcommand{\ee}{\end{enumerate}}

\newcommand{\bi}{\begin{itemize}}
\newcommand{\ei}{\end{itemize}}
\def\ds{\displaystyle}
\begin{document}
\setlength{\parindent}{0cm}
\renewcommand{\headrulewidth}{0pt}
\newcommand{\blank}[1]{\rule{#1}{0.75pt}}
\renewcommand{\d}{\displaystyle}
\vspace*{-0.7in}
\begin{center}
 {\large{ \sc{Section 5.2: Definite Integrals}}}
 \end{center}
%\begin{sc}
%\begin{center}
%Section 5.2: Definite Integrals and ``Area So Far''
%\end{center}
%\end{sc}

\section*{Definite Integrals and Areas ``under'' Curves}
\begin{enumerate}


\item  Estimate the area under $f(x) = x^2 - 2x$ on $[1,
3]$ with $n = 4$ using the 

  \begin{adjustbox}{valign=t,minipage={.65\textwidth}}
  %\begin{multicols}{2}{
      % make sure you added \usepackage{enumerate}
      %\vspace*{-0.45in}
     % \begin{enumerate}[(a)]
     (a) Right-hand endpoints 
     \vspace{1in}
     
     (b)  Left-hand endpoints %\hspace{2cm}
      \vspace{1in}
    %  \end{enumerate}}
  %\end{multicols}
  \end{adjustbox}
  %
  \begin{adjustbox}{valign=t,minipage={.25\textwidth}}
\begin{tikzpicture}[scale = 1.2]
\draw[<->,  thick](-.2,0) -- (3.2,0) node[right]{$x$};
\foreach \i in {0, ..., 3}{\draw (\i,.1) -- (\i, -.1) node[below]{$\i$};}
\draw[<->,  thick](0,-0.2) -- (0,3.2) node[above] {$y$};
\foreach \i in {-2, ..., 3}{\draw (.1, \i) -- ( -.1,\i) node[left]{$\i$};}
\draw[help lines] (0,-2) grid[step=.5] (3,3);
\draw[] (0,-2) grid[step=1] (3,3);
\draw[ultra thick] plot[smooth, domain=0:3] (\x, {\x*\x-2*\x});
\draw (2.1,1) node[above] {$f(x)$};
\end{tikzpicture}
\end{adjustbox}
%  \vspace{2in}


\begin{framed}
   \textbf{Definition of a Definite Integral} If $f$ is a function
   defined for $a \le x \le b$, we divide the interval $[a, b]$ into
   $n$ subintervals of equal width $\Delta x = (b-a)/n$. We let
   $x_0 = (a), x_1, x_2, \cdots, x_n =(b)$ be the endpoints of these
   subintervals and we let $x_1^*, x_2^*, \cdots, x_n^*$ be 
   \textbf{sample points}\footnotemark \ %\footnote{For example, we could choose our sample points to be right-hand endpoints, left-hand endpoints, midpoints,  a combination of these, or any other sample points in the interval that we choose!} 
    in these subintervals, so $x_i^*$ lies in
   the i-th subinterval $[x_{i-1}, x_i]$. Then the \textbf{definite
     integral of $f$ from $a$ to $b$} is

$$\d \int_a^b f(x) dx = \lim_{n \to \infty} \sum_{i=1}^n f(x_i^*)
\Delta x,$$
provided this limit exists and gives the same value for all possible
choices of sample points. If it does exist, we say that $f$ is
\textbf{integrable} on $[a, b]$.

\smallskip
${}^{1}$ \begin{footnotesize}For example, we could choose our sample points to be right-hand endpoints, left-hand endpoints, midpoints,  a combination of these, or any other sample points in the interval that we choose!\end{footnotesize}

\end{framed}



\item Consider again $f(x) = x^2 - 2x$ on the interval $[1,
3]$. Suppose that we are dividing the interval $[1,3]$ into $n$ subintervals.  (Think about your answers to \#1.)

\be

\item What is the length of each subinterval? \hrulefill
\vfill
\item What is the right-hand endpoint of the first subinterval? \hrulefill 
\vfill

What is the height of the first right-hand rectangle? \hrulefill
\vfill

\item What is the right-hand endpoint of the second subinterval? \hrulefill
\vfill

What is the height of the second right-hand rectangle? \hrulefill
\vfill

\item What is the right-hand endpoint of the third subinterval?\hrulefill 
\vfill

What is the height of the third right-hand rectangle? \hrulefill
\vfill

\item What is the right-hand endpoint of the $i^{\text{th}}$ rectangle? \hrulefill
\vfill


What is the height of the $i$-th right-hand rectangle? \hrulefill
\vfill

What is the {\bf area} of the $i$-th right-hand rectangle? \hrulefill
\vfill

\ee

\item Using your answers to the previous problem, write down a limit that equals $\d \int_{1}^{3}  x^2 - 2x \ dx$.

\vspace{1in}

%\ee
%\item  Find the area under $f(x)= (x^2 - 2x) dx$ on $[0,4]$ exactly. 

\item Write down a limit that equals $\d \int_{2}^{8} e^{x} \ dx$, using right-hand endpoints as your sample points.

\vspace{1in}

\begin{framed}
\emph{A definite integral represents the
{\bf signed} area under a curve (that is, the signed area between the curve and the $x$-axis). If a curve is above the $x$-axis
that area is \blank{1in}; if the curve is below the $x$-axis the area
is \blank{1in}.}
\end{framed}

%\newpage
\item  Evaluate the following definite integrals by drawing the function and interpreting the integral
in terms of areas. Shade in the area you are computing with the integral.

     
 \begin{tabular}{c c}%{\linewidth}{ X X X} 
   (a)   $\d \int_{-3}^3 (x - 1)\  dx = $\blank{.5in} & (b) $\d \int_0^4 \sqrt{16-x^2} \ dx $= \blank{.5in} \\  \tikz[scale = .5]{
\draw (-5, -5) grid (5,5);
\draw[<->, ultra thick] (0,-5.2) -- (0,5.2);
\draw[<->, ultra thick] (-5.2,0) -- (5.2,0);
}

& \tikz[scale = .5]{
\draw (-5, -5) grid (5,5);
\draw[<->, ultra thick] (0,-5.2) -- (0,5.2);
\draw[<->, ultra thick] (-5.2,0) -- (5.2,0);
}
\\

(c) $\d \int_{-3}^3 (2 + \sqrt{9-x^2} )\  dx$ = \blank{.5in}  & $\d \int_{-2}^3 ( |x| - 3 )\  dx$ = \blank{.5in}\\
   %

\tikz[scale = .5]{
\draw (-5, -5) grid (5,5);
\draw[<->, ultra thick] (0,-5.2) -- (0,5.2);
\draw[<->, ultra thick] (-5.2,0) -- (5.2,0);
}
&
\tikz[scale = .5]{
\draw (-5, -5) grid (5,5);
\draw[<->, ultra thick] (0,-5.2) -- (0,5.2);
\draw[<->, ultra thick] (-5.2,0) -- (5.2,0);
}
\end{tabular}

  
  
\newpage
\begin{adjustbox}{valign=t,minipage={.45\textwidth}}
\item The graph of $f$ is shown. Evaluate each integral
by interpreting it in terms of areas.
%\begin{multicols}{2}
  \begin{enumerate}[(a)]
  \item $\d \int_2^5 f(x) dx = $
\vskip0.5in
  \item $\d \int_5^9 f(x) dx = $
\vskip0.5in
  \item $\d \int_3^7 f(x) dx = $
  \end{enumerate}
%\columnbreak 
%  \begin{center}
%    \includegraphics[width=2.5in]{area}
%  \end{center}
%\end{multicols}
\end{adjustbox}
\begin{adjustbox}{valign=t,minipage={.45\textwidth}}
\begin{tikzpicture}[scale = .7]
\draw[<->,  thick](-1.2,0) -- (10.2,0) node[right]{$x$};
\foreach \i in {-1,1, ..., 10}{\draw (\i,.1) -- (\i, -.1) node[below]{$\i$};}
\draw[<->,  thick](0,-4.2) -- (0,4.2) node[above] {$y$};
\foreach \i in {-4, ..., 4}{\draw (.1, \i) -- ( -.1,\i) node[left]{$\i$};}
\draw[thin] (-1,-4) grid (10,4);
\draw[ultra thick] (0,1) -- (2,3) -- (3,3) -- (7,-3)--(9,-2);
\draw (1.5,2.8) node[above] {$f(x)$};
\end{tikzpicture}
\end{adjustbox}


\subsection*{Properties of the Definite Integral:}


%\begin{multicols}{2}
%\vspace*{-0.4in}

  \begin{itemize}
  \item $\d \int_a^b f(x) \ dx = $ \hrulefill %\vskip0.5in
  \item $ \d \int_a^a f(x)  \ dx = $ \hrulefill %\vskip0.5in
  \item $\d \int_a^b c \ dx = $ \hrulefill%\vskip0.5in
  \item $\d \int_a^b c f(x) \ dx = $ \hrulefill%\vskip0.5in
  \item $\d \int_a^b [f(x) \pm g(x)] dx =$ \hrulefill
  \item $\d \int_a^b f(x)  + \int_{b}^{c}f(x) \ dx = $ \hrulefill
  \item $\d \int_b^a f(x) \ dx = $ \hrulefill %\vskip0.5in
  \end{itemize}
%\end{multicols}


%\vfill
\item  Using the fact that $\d \int_0^1 x^2 dx = \frac 1 3$ and $\d \int_{1}^{2} x^{2}\ dx = \frac{7}{3}$,
evaluate the following using the properties of integrals. 

  \begin{multicols}{4}{
      % make sure you added \usepackage{enumerate}
      \vspace*{-0.45in}
      \begin{enumerate}[(a)]
      \item $\d \int_1^0 x^2 \ dx$
      \item $\d \int_0^1 5x^2\  dx$
      \item $\d \int_0^1 ( 4 + 3x^2)\ dx$
       \item $\d \int_0^2 x^{2}\ dx$. 
      \end{enumerate}}
  \end{multicols}
\vfill
\newpage



\ee

\end{document}
 \end

