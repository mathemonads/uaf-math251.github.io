\documentclass[11pt,fleqn]{article} 
\usepackage[margin=0.8in, head=0.8in]{geometry} 
\usepackage{amsmath, amssymb, amsthm}
\usepackage{fancyhdr} 
\usepackage{palatino, url, multicol}
\usepackage{graphicx, pgfplots} 
\usepackage[all]{xy}
\usepackage{polynom,tabularx} 
\usepackage{enumerate}
\usepackage{framed}
\usepackage{setspace}
\usepackage{array}
\usepackage{pgf,tikz}
\usepackage{mathrsfs}
\usetikzlibrary{arrows}

\usetikzlibrary{calc}

\pgfplotsset{compat=1.6}

\pgfplotsset{soldot/.style={color=blue,only marks,mark=*}} \pgfplotsset{holdot/.style={color=blue,fill=white,only marks,mark=*}}

\renewcommand{\headrulewidth}{0pt}
\newcommand{\blank}[1]{\rule{#1}{0.75pt}}
\newcommand{\bc}{\begin{center}}
\newcommand{\ec}{\end{center}}
\newcommand{\be}{\begin{enumerate}}
\newcommand{\ee}{\end{enumerate}}

\renewcommand{\d}{\displaystyle}



\pagestyle{fancy} 
%\lfoot{Uses a calculator}
\rfoot{Review: Chapters 3 \& 4}

\begin{document}

\vspace*{-0.7in}

\begin{center}
  \large
  \sc{Lecture Notes: Review of Chapters 3 \& 4}\\
\end{center}


\bc Summary of Topics \ec
Chapter 3\\
\begin{itemize}
	\item Recall Sections 1-6 involve derivative rules. This will \emph{not} be explicitly tested.
	\item Sections 7 and 8 focus on applications of the derivative in science and particularly to exponential growth and decay. Position, velocity and acceleration were again discussed. The overall emphasis is on \emph{interpretation} of the derivative in the context of an applied problem.
	\item Section 9 Related Rate Problems. In these problems you are always taking the derivative implicitly with respect to time and almost always seeking of find a rate of change at a particular instant.
	\item Section 10 Linear Approximations and Differentials. The crucial idea here is that the derivative can be used to estimate function-values or changes in function-values.
	\item Section11 we did not cover.
\end{itemize}

Chapter 4\\
\begin{itemize}
	\item Section 1 make a careful study of the ideas of local/absolute maximum/minimum and the difference between an extreme value and where it occurs.
	\item Section 4.2 The Mean Value Theorem. Know, roughly, what it says and be able to draw a picture.
	\item Section 4.3 discussed how the sign of $f'$ and $f''$ can tell us things about $f$ such as intervals on which $f$ is increasing, decreasing, concave up, concave down, local/absolute extreme values.
	\item Section 4.4 involved L'H\^{o}pital's Rule. Recall that before using this rule one should make sure it applies.
	\item Section 4.5 put a whole bunch of Calculus together to sketch a graph. In addition to topics from Section 1 and 2, we also included things like $x$- and $y$-intercepts, vertical and horizontal asymptotes, and the function's domain.
	\item Section 4.6 was not discussed.
	\item Section 4.7 involved Optimization. Recall that by this time we have a clear  understanding of how the  domain of the function may determine the techniques we use to determine the answer. 
	\item Section 4.8 will be discussed at the end of the semester and will not appear on this midterm.
	\item Section 4.9 involves antiderivatives.
\end{itemize}
%page 1
\newpage
\begin{enumerate}
%page 1
%linearization
\item Find the linearization of $f(x)=\sqrt{x}$ at $a=4$ and use it to estimate $\sqrt{4.1}$ and $\sqrt{3.8}$.
\vfill
\item Find the differential of $y=\sqrt{x}$ and use it to estimate how much $y$ will change as $x$ changes from $x=4$ to $x=4.1.$
\vfill
%lhop
\item Evaluate the following limits. Show your work. 
\be
\begin{multicols}{2}
\item $\d \lim_{x \to 0} \frac{1+x- e^x}{\sin x}$
\item  $\d \lim_{x \to\infty} x \ln (1+\frac{2}{x})$
\end{multicols}
\vspace{2.5in}
\ee
\newpage
\item Find the domain of the function $f(x)=\frac{\sin(5x)}{x^2+x}$ and identify any vertical or horizontal asymptotes. Justify your answers.\\

\vspace{2.5in}

%max and min on closed interval
\item $f(x)=(x-4)\sqrt[3]{x}=x^{4/3}-4x^{1/3}; \quad f'(x)=\frac{4(x-1)}{3x^{2/3}}; \quad f''(x) = \frac{4(x+2)}{9x^{5/3}}.$
\be
  \item Find the critical numbers of $f(x).$
  \vfill
   \item Find the open intervals on which the function is increasing or
    decreasing. \vfill
    
    \item Classify all critical points -- using the \textbf{first derivative test}.  \vfill
    \newpage
    From the other side: $f(x)=(x-4)\sqrt[3]{x}=x^{4/3}-4x^{1/3}; \quad f'(x)=\frac{4(x-1)}{3x^{2/3}}; \quad f''(x) = \frac{4(x+2)}{9x^{5/3}}.$
    \item Classify all critical points -- using the \textbf{second derivative test}.  \vfill
    
    

  \item Find the open intervals on which the function is concave up or
    concave down.  \vfill
    
  \item Find the inflection points.   \vfill
  
  \item Sketch the graph.
  \vfill
\ee
\ee
\end{document}
\newpage

%page 2
%related rate
\item A paper cup has the shape of a cone with height 10 cm and radius 3 cm (at the top). If water is poured into the cup at a rate of $2\: \text{cm}^3/\text{sec}$, how fast is the water level rising when the water is 5 cm deep? (Volume of a cone is: $V=(1/3) \pi r^2 h.$)
\vfill
\newpage
%page 5
\item Find the dimensions of the rectangle of maximum area that can be inscribed in an equilateral triangle of side 20 cm if one side of the rectangle likes on the base of the triangle.
\vfill
%Mean Value Theorem
\item 
\be
\item State the Mean Value Theorem and draw a picture to illustrate it.
\vspace{2in}
\item Determine whether the Mean Value Theorem applies
to $f(x) = x(x^2 - x - 2)$ on $[-1, 1]$. If it can be applied find all
numbers that satisfy the conclusion of the Mean Value Theorem. 
\ee
\vfill
\item Complete two iterations of Newton's Method to estimate a solution to $x^7+4=0$. Use $x_1=-1.$
\vfill
\end{enumerate}
\end{document}






