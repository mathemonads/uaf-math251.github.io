\documentclass[11pt,fleqn]{article} 
\usepackage[margin=0.8in, head=0.8in]{geometry} 
\usepackage{amsmath, amssymb, amsthm}
\usepackage{fancyhdr} 
\usepackage{palatino, url, multicol}
\usepackage{graphicx} 
\usepackage[all]{xy}
\usepackage{polynom} 
\usepackage{pdfsync}
\usepackage{enumerate}
\usepackage{framed}
\usepackage{setspace, adjustbox}
\usepackage{array%,tikz, pgfplots
}

\usepackage{tikz, pgfplots}
\usetikzlibrary{calc}
%\pgfplotsset{my style/.append style={axis x line=middle, axis y line=
%middle, xlabel={$x$}, ylabel={$y$}, axis equal }}
%
\pagestyle{fancy} 
\lfoot{UAF Calculus I}
\rfoot{3-9}


\newcommand{\be}{\begin{enumerate}}
\newcommand{\ee}{\end{enumerate}}

\newcommand{\bi}{\begin{itemize}}
\newcommand{\ei}{\end{itemize}}

\begin{document}
\setlength{\parindent}{0cm}
\renewcommand{\headrulewidth}{0pt}
\newcommand{\blank}[1]{\rule{#1}{0.75pt}}
\renewcommand{\d}{\displaystyle}
\vspace*{-0.7in}
\begin{center}
 {\large{ \sc{Section 3.9: Related Rates}}}
\end{center}
 	
There is a class of problems in calculus, known as related rate problems.
Here's the idea.  You know the rate of change (often with respect to time)
of one quantity, such as the volume of a spherical balloon.  
You want to know the rate of
change of some other related quantity (e.g. the radius of the balloon).  
Here are the steps you take to solve a problem like this:
\begin{enumerate}
\item Identify the quantity you already \fbox{know} a rate of change of (say, $V$, so 
you know $dV/dt$).
\item Identify the quantity you \fbox{want} a rate of change of (say, $r$, so you want $dr/dt$).
\item \fbox{Find an equation that relates} the two quantities ($V$ and $r$). This can be the hard part.  Drawing a picture can help.
\item Now take a derivative with respect to $t$ 
of both sides of the equation, treating both $V$ and $r$ as functions of $t$.
\item Substitute all known data into the result (typically $V$, $r$ and $dV/dt$) to determine $dr/dt$.
\end{enumerate}
We'll repeat this procedure with a bunch of examples.
\begin{enumerate}
\item Air is being pumped into a spherical balloon so
that its volume increases at a rate of 4.5 ft$^3$/min. How fast is the
radius of the balloon increasing when the diameter is 4 ft?
\vfill
\newpage
\item  Water runs into a conical tank at the rate of 9
ft$^3$/min. The tank stands point down and has a height of 10 ft and a
base radius of 5 ft. How fast is the water level rising when the water
is 6 ft deep?  
\vfill
\item  A street light is mounted at the top of a
10-ft-tall pole. A woman 5 ft tall walks away from the pole along a
straight path at a speed of 5 ft/s. How fast is the tip of her shadow moving when she is 40 ft from the pole?
\vfill
\newpage
\item A pebble dropped into a calm pond, causing ripples
in the form of circles. The radius $r$ of the outer ripple is
increasing at a constant rate of 1 foot per second. When the radius is
4 feet, at what rate is the area $A$ of the water disturbed changing?
\vfill
\item  A hot air balloon rising straight up from a level
field is tracked by a range finder 500 feet from the lift-off
point. At the moment the range finder's elevation angle is $\pi/4$,
the angle is increasing at the rate of 0.14 radians/min. How fast is
the balloon rising at that moment? 
\vfill
\newpage
\item  The standard 12 foot ladder rests against a vertical
wall. If the bottom of the ladder slides away from the wall at a rate
of 1ft/s, how fast is the top of the ladder sliding down the wall when
the bottom of the ladder is 6 ft from the wall? 
\vfill
\item  A police cruiser, approaching a right-angled
intersection from the north, is chasing a speeding car that has turned
the corner and is now moving straight east. When the cruiser is 0.6 mi
north of the intersection and the car is 0.8 mi to the east, the
police determine that the distance between them and the car they are
chasing is increasing at a rate of 20 mph. If the cruiser is moving at
60 mph at the instant of measurement, what is the speed of the car?
[Hint: You'll need to relate \textit{three} quantities here!]
\vfill

\end{enumerate}
\end{document}