
\documentclass[11pt,fleqn]{article} 
\usepackage[margin=0.8in, head=0.8in]{geometry} 
\usepackage{amsmath, amssymb, amsthm}
\usepackage{fancyhdr} 
\usepackage{palatino, url, multicol, tabularx}
\usepackage{graphicx} 
\usepackage[all]{xy}
\usepackage{polynom} 
\usepackage{pdfsync}
\usepackage{enumerate}
\usepackage{framed}
\usepackage{setspace, adjustbox}
\usepackage{array%,tikz, pgfplots
}

\usepackage{tikz, pgfplots}
\usetikzlibrary{calc}
%\pgfplotsset{my style/.append style={axis x line=middle, axis y line=
%middle, xlabel={$x$}, ylabel={$y$}, axis equal }}
%
\pagestyle{fancy} 
\lfoot{UAF Calculus I}
\rfoot{5-2}

\newcommand{\be}{\begin{enumerate}}
\newcommand{\ee}{\end{enumerate}}

\newcommand{\bi}{\begin{itemize}}
\newcommand{\ei}{\end{itemize}}
\def\ds{\displaystyle}
\begin{document}
\setlength{\parindent}{0cm}
\renewcommand{\headrulewidth}{0pt}
\newcommand{\blank}[1]{\rule{#1}{0.75pt}}
\renewcommand{\d}{\displaystyle}
\vspace*{-0.7in}

%\begin{sc}
%\begin{center}
%Section 5.2: Definite Integrals and ``Area So Far''
%\end{center}
%\end{sc}


%\newpage
\begin{center}
 {\large{ \sc{Section 5.2 - 3: ``Area So Far'' functions}}}
 \end{center}
\section*{``Area So Far'' functions}

\be

\item Let $f(x)$ be given by the graph below and define $\displaystyle{A(x) = \int_0^x f(t)dt}$.

%\begin{figure}[ht]
%\begin{center}
%\includegraphics{AntiAreaGraph1}
%\end{center}
%\end{figure}
\begin{center}
\begin{tikzpicture}
\draw[<->, thick](-.2,0) -- (5.2,0);
\foreach \i in {0,1, ..., 5}{\draw (\i,.1) -- (\i, -.1) node[below]{$\i$};}
\draw[<->,thick](0,-.2) -- (0,3.2);
\foreach \i in {0,1, ..., 3}{\draw (.1, \i) -- ( -.1,\i) node[left]{$\i$};}
\draw[thin] (0,0) grid (5,3);
\draw[ultra thick] (0,2) -- (2,2) -- (3,3) -- (5,1);
\draw (1.5,2) node[above] {$f(x)$};
\end{tikzpicture}
\end{center}

Compute the following using the graph. Hint: $A(1) = \int_{0}^{1} f(x) \ dx$, which calculates the area accumulated under the graph from $x = 0$ to $x = 1$.

\renewcommand{\baselinestretch}{3}
\begin{multicols}{2}

\renewcommand{\baselinestretch}{3}

$A(1) =$ \hrulefill

\vfill
$A(2) = $\hrulefill
\vfill

$A(3) =$ \hrulefill
\vfill

$A(4) = $\hrulefill
\vfill

$A(5) = $\hrulefill
\vfill


$f(1) = $\hrulefill
\vfill

$f(2) = $\hrulefill
\vfill

$f(3) = $\hrulefill
\vfill

$f(4) = $\hrulefill

$f(5) = $\hrulefill
\end{multicols}

The $x$-value in the interval $[0,5]$  at which $A(x)$ attains its maximum is \hrulefill
\vfill

The maximum value of $A(x)$ on $[0,5]$ is \hrulefill
\vfill

The  $x$-value in the interval $[0,5]$  at which $f(x)$ attains its maximum is \hrulefill
\vfill

The maximum value of $f(x)$ on $[0,5]$ is \hrulefill
\vfill

What can you say about the {\bf rate of change} of $A(x)$? 
\vfill

\newpage


\item %A toy car is travelling on a straight track. Its velocity $v(t)$, in meters per second, is given by the graph below. Define $s(t)$ to be the position of the car in meters, and suppose that $s(0) = 0$. 

A toy car is travelling on a straight track. Its velocity $v(t)$, in meters per second, is given by the graph below. Define $s(t)$ to be the position of the car in meters, and suppose that $s(0) = 0$. Note that $s(t)= \displaystyle{\int_{0}^{t}} v(x) \ dx$. (Here, $x$ is called the ``dummy variable of integration''.)
%
%\begin{figure}[ht]
%\begin{center}
%\includegraphics{AntiAreaGraph2}
%\end{center}
%\end{figure}

\begin{center}
\begin{tikzpicture}
\draw[<->, thick](-.2,0) -- (6.2,0) node[right] {$t$};
\foreach \i in {0, ..., 6}{\draw (\i,.1) -- (\i, -.1) node[below]{$\i$};}
\draw[<->, thick](0,-1.2) -- (0,3.2);
\foreach \i in {-1, ..., 3}{\draw (.1, \i) -- ( -.1,\i) node[left]{$\i$};}
\draw[thin] (0,-1) grid (6,3);
\draw[ultra thick] (0,2) -- (1,2) -- (4,-1) -- (6,3);
\draw (1.5,2) node[above] {$v(t)$};
\end{tikzpicture}
\end{center}

Compute the following:
\begin{multicols}{3}
$s(2) = $ \hrulefill

$v(2) = $ \hrulefill

$s(4) = $ \hrulefill

$v(4) = $ \hrulefill

$s(6) = $ \hrulefill

$v(6) = $ \hrulefill

\end{multicols}

The $t$-value in the interval $[0,6]$  at which $s(t)$ attains its maximum is \hrulefill
\vfill

The maximum value of $s(t)$ on $[0,6]$ is \hrulefill
\vfill

The  $t$-value in the interval $[0,6]$  at which $s(t)$ attains its minimum is \hrulefill
\vfill

The minimum value of $s(t)$ on $[0,6]$ is \hrulefill
\vfill


The $t$-value in the interval $[0,6]$  at which $v(t)$ attains its maximum is \hrulefill
\vfill

The maximum value of $v(t)$ on $[0,6]$ is \hrulefill
\vfill

The  $t$-value in the interval $[0,6]$  at which $v(t)$ attains its minimum is \hrulefill
\vfill

The minimum value of $v(t)$ on $[0,6]$ is \hrulefill
\vfill

Describe the position of the car over the 6 seconds. \hrulefill
\vfill
\hrulefill
\vfill
\hrulefill
\vfill

Describe the velocity of the car over the 6 seconds. \hrulefill
\vfill
\hrulefill
\vfill
\hrulefill
\vfill

\ee

%
%\newpage
%\item An artist wants to make a figure consisting of the region between the curve $y = x^2$ and the $x$-axis for $0 \leq x \leq 3$.
%
%\begin{figure}[ht]
%\begin{center}
%\scalebox{.8}{\includegraphics{AntiAreaGraph3}}
%\end{center}
%\end{figure}
%
%\be 
%
%\item Where should the artist divide the region with a vertical line so that each piece has the same area? (Hint: think about some definite integrals)
%
%\item Where should the artist divide the region with two vertical lines to get three pieces which each have the same area?
%\ee
%
%

\end{document}
