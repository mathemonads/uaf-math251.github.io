
\documentclass[12pt]{article}

% Layout.
\usepackage[top=1.0in, bottom=0.75in, left=1in, right=1in, headheight=1.0in, headsep=0pt]{geometry}

% Fonts.
\usepackage{mathptmx}
\usepackage[scaled=0.86]{helvet}
\renewcommand{\emph}[1]{\textsf{\textbf{#1}}}

% TiKZ.
\usepackage{tikz, pgfplots}
\usetikzlibrary{calc}
\pgfplotsset{my style/.append style={axis x line=middle, axis y line=
middle, xlabel={$x$}, ylabel={$y$}, axis equal }}

% Misc packages.
\usepackage{amsmath,amssymb,latexsym}
\usepackage{graphicx}
\usepackage{array}
\usepackage{xcolor}
\usepackage{multicol}
\usepackage{adjustbox}


% Commands to set various header/footer components.
\makeatletter
\def\doctitle#1{\gdef\@doctitle{#1}}
\doctitle{Use {\tt\textbackslash doctitle\{MY LABEL\}}.}
\def\docdate#1{\gdef\@docdate{#1}}
\docdate{Use {\tt\textbackslash docdate\{MY DATE\}}.}
\def\doccourse#1{\gdef\@doccourse{#1}}
\let\@doccourse\@empty
\def\docscoring#1{\gdef\@docscoring{#1}}
\let\@docscoring\@empty
\def\docversion#1{\gdef\@docversion{#1}}
\let\@docversion\@empty
\makeatother

% Headers and footers layout.
\makeatletter
\usepackage{fancyhdr}
\pagestyle{fancy}
\fancyhf{} % Clears all headers/footers.
\lhead{\emph{\@doctitle\hfill\@docdate}
\ifnum \value{page} > 1\relax\else\\
\emph{Name: \rule{3.5in}{1pt}\hfill \@docscoring}
\\
%\emph{Circle one: \quad Rhodes (F01) \hskip 1ex\rule{1pt}{9pt}\hskip 1ex Bueler (F02)}
\fi}

\rfoot{\emph{\@docversion}}
\lfoot{\emph{\@doccourse}}
\cfoot{\emph{\thepage}}
\renewcommand{\headrulewidth}{0pt}%
\makeatother

% Paragraph spacing
\parindent 0pt
\parskip 6pt plus 1pt

% A problem is a section-like command. Use \problem{5} to
% start a problem worth 5 points.
\newcounter{probcount}
\newcounter{subprobcount}
\setcounter{probcount}{0}
\newcommand{\problem}[1]{%
\par
\addvspace{4pt}%
\setcounter{subprobcount}{0}%
\stepcounter{probcount}%
\makebox[0pt][r]{\emph{\arabic{probcount}.}\hskip1ex}\emph{[#1 points]}\hskip1ex}
\newcommand{\thesubproblem}{\emph{\alph{subprobcount}.}}

% Subproblems are an enumerate-like environment with a consistent
% numbering scheme. 
% Use \begin{subproblems}\item...\item...\end{subproblems}
\newenvironment{subproblems}{%
\begin{enumerate}%
\setcounter{enumi}{\value{subprobcount}}%
\renewcommand{\theenumi}{\emph{\alph{enumi}}}}%
{\setcounter{subprobcount}{\value{enumi}}\end{enumerate}}

% Blanks for answers in normal and math mode.
\newcommand{\blank}[1]{\rule{#1}{0.75pt}}
\newcommand{\mblank}[1]{\underline{\hspace{#1}}}
\def\emptybox(#1,#2){\framebox{\parbox[c][#2]{#1}{\rule{0pt}{0pt}}}}

% Misc.
\renewcommand{\d}{\displaystyle}
\newcommand{\ds}{\displaystyle}
\def\bc{\begin{center}}
\def\ec{\end{center}}

\newcommand{\ans}[1][2]{ \ \rule{#1 in}{.5 pt} \ }


\doctitle{Math 251: Quiz 8}
\docdate{November 5, 2019}
\doccourse{UAF Calculus I}
\docversion{v-1}
\docscoring{{\Large \strut}\blank{0.8in} / 25}

\begin{document}
25 points possible.  No aids (book, calculator, etc.) are permitted.  You need not simplify, but show all work and use proper notation for full credit.

% Typical 4.3 or 4.5 problem
\problem{9}  The function $j(x)$ and its first two derivatives are given below. Use them to answer parts (a)-(d).\\
$$\ds{j(x)=\frac{(x+1)^2}{x^2+1}, \hspace{.5in}j'(x)=\frac{-2(x-1)(x+1)}{(x^2+1)^2}, \hspace{.5in}j''(x)=\frac{4x(x^3+3)}{(x^2+1)^3}}$$ 
\begin{subproblems}
	
	\item Does $j(x)$ have any vertical asymptotes? Justify your answer.
	\vfill
	\item Does $j(x)$ have any horizontal asymptotes? Justify your answer.
	\vfill
	\item Determine the intervals on which $j(x)$ is increasing or decreasing. Show your work to receive credit.
	\vspace{1.5in}
	\item Identify where $j(x)$ has any local minimums or local maximums.
	\vfill
\end{subproblems}

% like 4.1 #48
\problem{8} Find the limit.
	\begin{subproblems}
	\item $\ds{\lim_{t \to 0}\frac{e^{17t}-1}{\sin(2t)}}$
	\vfill
	\newpage
	\item $\ds{\lim_{x \to 0^+}\left(\frac{1}{x}-\frac{1}{e^{x}-1}\right)}$
	\vfill
	\end{subproblems}
    


\problem{8} On the axes below, sketch the graph of a function that satisfies \emph{all} of the given conditions. Label on your sketch any local maximums, any local minimums, and any inflection points.
	\begin{subproblems}
	\item $k(x)$ is continuous and differentiable for all real numbers.
	\item $k(0)=2$
	\item The table below gives information about the sign of first derivative of $k(x).$
	\begin{center}
	\begin{tabular}{c || c | c | c | c | c }
	$x$ & $-\infty < x < -4$ & $x=-4$ & $-4 < x < 0$& $x=0$& $0 < x < \infty$ \\
	\hline
	$k'(x)$& $-$ & 0 & $+$ & $0$ & $+$ \\
	\end{tabular}
	\end{center}
	\item The table below gives information about the sign of second derivative of $k(x).$
	\begin{center}
	\begin{tabular}{c || c | c | c | c | c }
	$x$ & $-\infty < x < -1$ & $x=-1$ & $-1 < x < 0$& $x=0$& $0 < x < \infty$ \\
	\hline
	$k''(x)$& $+$ & 0 & $-$ & $0$ & $+$ \\
	\end{tabular}
	\end{center}
	\begin{tikzpicture}{scale=.7}
	\draw (-6,0) -- (4,0); \draw (0,-3) -- (0,5); 
	\foreach \i in {-4,-2,2}{ \draw (\i,-.2) -- (\i, .2); \node at (\i,-0.4){\i};}
	\draw (-0.2,2) -- (0.2,2); \node at (-0.4,2){2};
	\end{tikzpicture}
	\end{subproblems}
\end{document}