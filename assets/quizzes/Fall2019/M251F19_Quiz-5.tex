% !TEX TS-program = pdflatexmk
\documentclass[12pt]{article}

% Layout.
\usepackage[top=1in, bottom=0.75in, left=1in, right=1in, headheight=1in, headsep=6pt]{geometry}

% Fonts.
\usepackage{mathptmx}
\usepackage[scaled=0.86]{helvet}
\renewcommand{\emph}[1]{\textsf{\textbf{#1}}}

% TiKZ.
\usepackage{tikz, pgfplots}
\usetikzlibrary{calc}
\pgfplotsset{my style/.append style={axis x line=middle, axis y line=
middle, xlabel={$x$}, ylabel={$y$}, axis equal }}

% Misc packages.
\usepackage{amsmath,amssymb,latexsym}
\usepackage{graphicx}
\usepackage{array}
\usepackage{xcolor}
\usepackage{multicol}

% Commands to set various header/footer components.
\makeatletter
\def\doctitle#1{\gdef\@doctitle{#1}}
\doctitle{Use {\tt\textbackslash doctitle\{MY LABEL\}}.}
\def\docdate#1{\gdef\@docdate{#1}}
\docdate{Use {\tt\textbackslash docdate\{MY DATE\}}.}
\def\doccourse#1{\gdef\@doccourse{#1}}
\let\@doccourse\@empty
\def\docscoring#1{\gdef\@docscoring{#1}}
\let\@docscoring\@empty
\def\docversion#1{\gdef\@docversion{#1}}
\let\@docversion\@empty
\makeatother

% Headers and footers layout.
\makeatletter
\usepackage{fancyhdr}
\pagestyle{fancy}
\fancyhf{} % Clears all headers/footers.
\lhead{\baselineskip 30pt
\emph{\@doctitle\hfill\@docdate}
\ifnum \value{page} > 1\relax\else\\
\emph{Name: \rule{3.5in}{1pt}\hfill \@docscoring}\fi}
\rfoot{\emph{\@docversion}}
\lfoot{\emph{\@doccourse}}
\cfoot{\emph{\thepage}}
\renewcommand{\headrulewidth}{0pt}%
\makeatother

% Paragraph spacing
\parindent 0pt
\parskip 6pt plus 1pt

% A problem is a section-like command. Use \problem{5} to
% start a problem worth 5 points.
\newcounter{probcount}
\newcounter{subprobcount}
\setcounter{probcount}{0}
\newcommand{\problem}[1]{%
\par
\addvspace{4pt}%
\setcounter{subprobcount}{0}%
\stepcounter{probcount}%
\makebox[0pt][r]{\emph{\arabic{probcount}.}\hskip1ex}\emph{[#1 points]}\hskip1ex}
\newcommand{\thesubproblem}{\emph{\alph{subprobcount}.}}

% Subproblems are an enumerate-like environment with a consistent
% numbering scheme. 
% Use \begin{subproblems}\item...\item...\end{subproblems}
\newenvironment{subproblems}{%
\begin{enumerate}%
\setcounter{enumi}{\value{subprobcount}}%
\renewcommand{\theenumi}{\emph{\alph{enumi}}}}%
{\setcounter{subprobcount}{\value{enumi}}\end{enumerate}}

% Blanks for answers in normal and math mode.
\newcommand{\blank}[1]{\rule{#1}{0.75pt}}
\newcommand{\mblank}[1]{\underline{\hspace{#1}}}
\def\emptybox(#1,#2){\framebox{\parbox[c][#2]{#1}{\rule{0pt}{0pt}}}}

% Misc.
\renewcommand{\d}{\displaystyle}
\newcommand{\ds}{\displaystyle}
\def\bc{\begin{center}}
\def\ec{\end{center}}


\doctitle{Math 251: Quiz 5}
\docdate{Oct 5, 2019}
\doccourse{UAF Calculus I}
\docversion{v-1}
\docscoring{\blank{0.8in} / 25}
\begin{document}
There are 25 points possible on this quiz. No aids (book, calculator, etc.)
are permitted.  Show all work for full credit.


%\begin{enumerate}
%\item 
\problem{15} Find the derivatives of each of the following. You do not need to simplify your answer.


\begin{subproblems}

\item $\ds h(\theta) = e^{2}\sec(\theta) + \cot(\theta)$
\vfill

\item $\ds y = \cos({5x^{2}})$
\vfill

\item $\ds f(x) = \frac{\tan(x)}{x - 3\sin(x)}$ 
\vfill

\item $\ds f(q) = q^{3} e^{5q+6}$
\vfill

\item $\ds k(t) = (\sqrt[5]{t} - 7t+3)^{5}$
\vfill

\end{subproblems}
%4-5 derivatives easy to hard with chain

%+ two applications


\newpage
%% Do you understand sine and cosine and tangents.
\problem{4} Find an $x$-value such that the function $f(x)=2x+\cos(4x)$ has a horizontal tangent line. (You do not have to find \textit{every} value. Simply find one.) 

\vspace{2.5in}	

%Spring 2018 length of day problem

%\problem{4}
%A sinusoidal model\footnote{Sine functions actually turn out to be not very good at modeling the number of hours of daylight at high latitudes!} for the number of daylight hours in Fairbanks, Alaska is given by the function \[
%L(t) = 9.06\sin\left(\frac{2\pi}{365}(x - 80)\right)+12.76.
%\]
%where $L$ is measured in hours and $t$ is measured in days, with $t=0$
%representing January 1.
%\begin{subproblems}
%\item Compute $L'(t)$.
%\vskip 0pt plus 0.5fill
%\item October 15 corresponds to $t = 287$. Suppose you have computed $L'(287)\approx -0.1$.  Interpret
%what this means in precise language that your non-quantitatively-inclined friends could nevertheless understand.  Your answer must include units for full credit.  
%\vskip 0pt plus 0.5fill
%\end{subproblems}

%find the acceleration problem

\problem{6} In a certain experiment involving bacteria, the number $N$ of bacteria in a culture after $t$ days is modeled by the function
\[ N(t) = 900\left(1+ \frac{3}{(t^{2}+1)^{2}}\right).\]
\begin{subproblems}

\item How many bacteria are in the culture at the beginning of the experiment?
\vfill

\item Compute $N'(t)$. (You do not need to simplify, but you may if you choose.)

\vfill

\item After one day, is the number of bacteria in the culture {\bf increasing} or {\bf decreasing}, and how do you know? (Justify your answer; an answer with no justification will receive no credit.)

\vfill
\end{subproblems}

%\end{enumerate}

\end{document}