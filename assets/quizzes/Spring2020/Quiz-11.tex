\documentclass[12pt]{article}

% Layout.
\usepackage[top=1.2in, bottom=0.75in, left=1in, right=1in, headheight=1.0in, headsep=0pt]{geometry}

% Fonts.
\usepackage{mathptmx}
\usepackage[scaled=0.86]{helvet}
\renewcommand{\emph}[1]{\textsf{\textbf{#1}}}

% TiKZ.
\usepackage{tikz, pgfplots}
\usetikzlibrary{calc}
\pgfplotsset{my style/.append style={axis x line=middle, axis y line=middle, xlabel={$x$}, ylabel={$y$}}}

% Misc packages.
\usepackage{amsmath,amssymb,latexsym}
\usepackage{graphicx}
\usepackage{array}
\usepackage{xcolor}
\usepackage{multicol}

% Commands to set various header/footer components.
\makeatletter
\def\doctitle#1{\gdef\@doctitle{#1}}
\doctitle{Use {\tt\textbackslash doctitle\{MY LABEL\}}.}
\def\docdate#1{\gdef\@docdate{#1}}
\docdate{Use {\tt\textbackslash docdate\{MY DATE\}}.}
\def\doccourse#1{\gdef\@doccourse{#1}}
\let\@doccourse\@empty
\def\docscoring#1{\gdef\@docscoring{#1}}
\let\@docscoring\@empty
\def\docversion#1{\gdef\@docversion{#1}}
\let\@docversion\@empty
\makeatother

% Headers and footers layout.
\makeatletter
\usepackage{fancyhdr}
\pagestyle{fancy}
\fancyhf{} % Clears all headers/footers.
\lhead{\emph{\@doctitle\hfill\@docdate}
\ifnum \value{page} > 1\relax\else\\
\emph{Name: \rule{3.5in}{1pt}\hfill \@docscoring}
\\
%\emph{Circle one: \quad Faudree (F01) \hskip 1ex\rule{1pt}{9pt}\hskip 1ex Bueler (F02) \hskip 1ex\rule{1pt}{9pt}\hskip 1ex VanSpronsen (UX1)}
\fi}

\rfoot{\emph{\@docversion}}
\lfoot{\emph{\@doccourse}}
\cfoot{\emph{\thepage}}
\renewcommand{\headrulewidth}{0pt}%
\makeatother

% Paragraph spacing
\parindent 0pt
\parskip 6pt plus 1pt

% A problem is a section-like command. Use \problem{5} for a problem worth 5 points.
\newcounter{probcount}
\newcounter{subprobcount}
\setcounter{probcount}{0}
\newcommand{\problem}[1]{%
\par
\addvspace{4pt}%
\setcounter{subprobcount}{0}%
\stepcounter{probcount}%
\makebox[0pt][r]{\emph{\arabic{probcount}.}\hskip1ex}\emph{[#1 points]}\hskip1ex}
\newcommand{\thesubproblem}{\emph{\alph{subprobcount}.}}

% Subproblems are an enumerate-like environment with a consistent
% numbering scheme. 
% Use \begin{subproblems}\item...\item...\end{subproblems}
\newenvironment{subproblems}{%
\begin{enumerate}%
\setcounter{enumi}{\value{subprobcount}}%
\renewcommand{\theenumi}{\emph{\alph{enumi}}}}%
{\setcounter{subprobcount}{\value{enumi}}\end{enumerate}}

% Blanks for answers in normal and math mode.
\newcommand{\blank}[1]{\rule{#1}{0.75pt}}
\newcommand{\mblank}[1]{\underline{\hspace{#1}}}
\def\emptybox(#1,#2){\framebox{\parbox[c][#2]{#1}{\rule{0pt}{0pt}}}}

% Misc.
\renewcommand{\d}{\displaystyle}
\newcommand{\ds}{\displaystyle}


\doctitle{\textcolor{red}{OPTIONAL} Math 251: Quiz 11}
\docdate{Due 5:00PM 24 April, 2020}
\doccourse{UAF Calculus I}
%\docversion{v-1}
\docscoring{{\LARGE \strut}\blank{0.8in} / 16}

\begin{document}
Each problem is worth 2 points: 1 for work and 1 for the answer. Your work should be organized. Your answer should be in a \fbox{box}. This quiz is an excellent measure of readiness for Calculus II. You \emph{should} be able to complete this in 30 minutes or less. To encourage you to pay attention to time, we have added blanks for that, too. \textcolor{red}{If you use aids of any kind you are completely missing the point of this exercise.}\\

\textbf{Time Started:}\underline{\hspace{1in}} \hspace{.5in} \textbf{Time Started:}\underline{\hspace{1in}}\\

Evaluate the indefinite and definite integrals.\\

%baby sub
\problem{2} $\displaystyle{\int 5 \sin\left(\frac{\pi}{2}\theta\right) \: d \theta}$
\vfill

%baby sub + other term
\problem{2} $\displaystyle{\int 3x-e^{3x} \: dx}$
\vfill

%u as denominator
\problem{2} $\displaystyle{\int \frac{1}{6-7t}dt}$
\vfill

%something raised to a power
\problem{2} $\displaystyle{\int \frac{ax}{\sqrt{1-ax^2}} dx}$
\vfill
\newpage


\problem{2} $\displaystyle{\int_1^3 te^{t^2}dt}$
\vfill

\problem{2} $\displaystyle{\int_0^{\pi/4} \cos(2t)\left(1+ \sin(2t)\right)^2 \: dt}$
\vfill

\problem{2} $\displaystyle{\int_0^1 \frac{1+e^x}{x+e^x} dx}$
\vfill

\problem{2} $\displaystyle{\int \frac{dx}{1+16x^2}}$
\vfill



\end{document}